%
% 	Conclusion
%
This paper presents a novel approach for model-based acceptance testing. It introduces a minimal set of structural and behavioural models that are normally produced during the development process, and also are intuitive and tangible for programmers and modellers. The domain model, represented as a class diagram or EMF model, is used as the structural model. Use cases, described by the provided DSLs, are used as the behavioural model. A use case description basically defines the pre-/post-conditions, and possibly the data for the use case.
Our approach employs static analysis for automatic test generation and execution. All the models are automatically translated into Alloy specifications which are solved by the Alloy Analyser resulted in valid correct execution traces for the SUT.

The proposal was evaluated throughout a case study on a medium-sized business application and analysing the result based on different coverage criteria, namely line coverage, use case coverage, and data coverage. We also applied mutation testing to assess the quality and effectiveness of the generated test cases. The study demonstrates that the framework provides acceptable line coverage (90\%) and mutation score of 72\%. In addition to implementation errors, the framework would help in identifying errors in analysis and requirements specification.

As for future work, we plan to improve the introduced DSLs to be more concise and provide more modelling features. Moreover, we plan to enhance model transformations to have optimised Alloy models in order to reduce test execution times. Finally, we would like to work on techniques to efficiently deal with evolution in test models.


