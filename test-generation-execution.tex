%
% 	Farmework in Action: Test Generation and Execution
%
\label{sec:test-generation-execution}
The previous section introduces the required models and explains how to produce them. This section explains the test generation and execution workflow. The testing framework has been built upon Alloy~\cite{Jackson2002} and the Alloy Analyser (AA)~\cite{Jackson2000} with the aim of testing a system at the use case level (i.e.,\ acceptance testing). AA provides automatic analysis of Alloy specifications by generating instances that satisfy the constraints expressed in the specification.

The key idea behind the framework is to use Alloy to express the structural and behavioural model of the system specifying the use cases, the invariants of inputs and outputs, and pre-/post-conditions of executing use cases on the system under test. In this context, all the test models, introduced in the previous section, are automatically translated into Alloy specifications which are then used by AA to automatically generate, for a given scope, all non-isomorphic~\cite{Shlyakhter2007} instances for that specification. Next, the testing framework translates these instances to real execution paths and concrete inputs, which form the executable test cases for the SUT.

In the following, we first describe the basics of the Alloy specification language and the Alloy Analyser; details can be found in~\cite{Jackson2000,Jackson2002,Jackson2012}. Then, we explain the model transformations into Alloy.

\subsection{Alloy}
\label{sec:test-generation-execution-alloy}
Alloy is a strongly typed language that assumes a universe of atoms partitioned into subsets, each of which has a basic type. An Alloy specification is a sequence of paragraphs that can be of two kinds: signatures, used for construction of new \textit{types}, and a variety of formula paragraphs, used to record \textit{constraints}. 

A \textit{signature} paragraph represents a basic (or uninterpreted) type and introduces an independent top-level set. A signature declaration may include a collection of relations (that are called \textit{fields}) along with the types of the fields and constraints on their values; a \textit{field} represents a set relation between the signatures. \textit{Formula} paragraphs are formed from Alloy expressions, specifying constraints of the desired solutions, such as \textit{facts} and \textit{assertions}. A \texttt{fact} is a formula that takes no arguments and need not be invoked explicitly; it is always true. An assertion (\texttt{assert}) is a formula whose correctness needs to be checked, assuming the facts in the model. 

The Alloy Analyser~\cite{Jackson2000} is an automatic tool for analysing models created in Alloy. Given a formula and a \textit{scope}--a bound on the number of atoms in the universe--the analyser determines whether there exists a model of the formula (that is, an assignment of values to the sets and relations that makes the formula true) that uses no more atoms than the scope permits, and if so, returns it. Since first order logic is undecidable, the analyser limits its analysis to a finite scope. The analysis is based on a translation to a boolean satisfaction problem, and gains its power by exploiting state-of-the-art SAT solvers. The models of formulae are termed \textit{instances} or \textit{solutions}. Each atom is in a signature and the relations between atoms are instances of their fields.

\subsection{Generated Alloy Model}
\label{sec:test-generation-execution-alloy-models}
The test models described in Section~\section{xxx} are automatically translated into Alloy specifications consisting of six \texttt{modules}, that are defined in separate Alloy files (.als). These modules and their dependencies are shown in Fig.~\ref{fig:alloy-specifications}, and include:

\begin{enumerate}
	\item \texttt{system\_def\_auto}: translation of the domain structural model (EMF diagram),
	\item \texttt{operations\_def\_auto}: translation of the system behavioural model (use case descriptions) and definition of the labels,
	\item \texttt{rules\_def\_auto}: translation of the object labels rules,
	\item \texttt{init\_def\_auto}: translation of the initial state model, 
	\item \texttt{input\_def\_auto}: translation of the input partitionings, and
	\item \texttt{main}: the main module specifying the test goals and the entry point for the static analysis (i.e.,\ the starting predicate).
\end{enumerate}

\begin{figure}[h]
\centering
\includegraphics[width=0.4\textwidth]{../Figures/alloy-specifications.png}
\caption{Generated Alloy specifications and their dependencies.}
\label{fig:alloy-specifications}
\end{figure}

The rest of this section describes the details of the model transformation for each module.

\subsection{Structure Model}
\label{sec:alloy-structure}
The EMF model of the system is automatically translated into an Alloy specification, using the Xpand\footnote{Xpand – \url{http://eclipse.org/modeling/m2t/?project=xpand}} model-to-text transformation language. Fig.~\ref{fig:library-alloy-structure} shows the automatically generated Alloy specification for the structure of the example library system. Each class is translated into a signature. Each association is represented by a field in the respective signature (lines 9-11 and lines 17-19). The multiplicity\todo{is that the right word?} of a relation affects the definition of the fields. Abstract signatures are used for modelling abstract classes. Class inheritance is implemented using the inheritance concept in Alloy.

\begin{figure}[h]
\centering
\includegraphics[width=0.5\textwidth]{../Figures/library-alloy-structure.png}
\caption{Alloy specification of the structure of the example library system.}
\label{fig:library-alloy-structure}
\end{figure}

%\subsubsection{State}
%\label{sec:alloy-state}
Alloy does not have a built-in notion of mutable state; it does not allow changes to the relations, whereas, in the context of testing, the domain model changes with the execution of use cases. Consequently, our approach models the concept of mutable state  explicitly. A typical way to deal with this is defining a \texttt{state} signature and all the relations that would change during test executions are placed in a relation with an atom of \texttt{state}. The transformation adds the extension \texttt{-> State} at the end of the changing relations.

The \texttt{System} signature represents the system under test and tracks the objects that in the domain model at any time. The transformation defines a field for each concrete class in the domain model (e.g.,\ Fig.~\ref{fig:library-alloy-structure}, lines 1-4).

Object labels are modelled by the \texttt{DomainObjState} abstract signature, such that for each domain class, the transformation adds an abstract signature that extends \texttt{DomainObjState} and shows the labels for that class (e.g.,\ lines 15-19). Additionally, all the signatures for domain classes inherit from the \texttt{DomainObj} abstract signature which contains the field defining the relation between atoms and their labels.

In order to constrain the state space objects not in the domain model cannot participate in relations and can not have label atoms. This is specified as a \texttt{fact} at the end of the structural Alloy specification. The fact for the example library system is depicted in Fig.~\ref{fig:library-alloy-structure-fact}.

\begin{figure}[h]
\centering
\includegraphics[width=0.4\textwidth]{../Figures/library-alloy-structure-fact.png}
\caption{The \texttt{fact} for confining the state space in the structure model of the example library system.}
\label{fig:library-alloy-structure-fact}
\end{figure}


\subsection{Operations: Use Cases}
\label{sec:alloy-usecases}
The next step is to translate the use case descriptions into Alloy. Each use case is translated into a corresponding signature and a predicate. First, the signature is used for tracing the execution flow: a solution does not show which predicate in the specification has resulted in the solution. The test driver needs to know which predicate has been selected at any state. Using \textit{conjunction}, an atom of the signature is attached to each predicate, so that it is possible to trace the execution flow. The use case signature contains fields corresponding to its parameters. Fig.~\ref{fig:library-alloy-return-book-signature} shows the signature for the ``Return Book" use case. Lines 5-8 restrict the solution space.

\begin{figure}[h]
\centering
\includegraphics[width=0.5\textwidth]{../Figures/library-alloy-return-book-signature.png}
\caption{The signature for the ``Return Book" use case.}
\label{fig:library-alloy-return-book-signature}
\end{figure}

Each use case predicate takes atoms representing the system state before and after the use case is executed. An atom of the signature of the use case provides access to the inputs. This predicate essentially defines the constraints for creating the subsequent state. Fig.~\ref{fig:library-alloy-return-book-predicate} shows the predicate for ``Return Book". This predicate consists of the following parts: defining the domain of the input parameters (lines 2-5), defining pre-conditions (lines 6-10), specifying the frame conditions (lines 11-20), and defining the post-conditions (lines 21-24). \todo{move these to figure caption}

\begin{figure}[h]
\centering
\includegraphics[width=0.5\textwidth]{../Figures/library-alloy-return-book-predicate.png}
\caption{The predicate for the ``Return Book" use case.}
\label{fig:library-alloy-return-book-predicate}
\end{figure}

The predicate of a use case may also contains 1) a part for defining variables (using the \textbf{let} structure) and 2) a part for creating new objects, when new objects are added to the domain model. An example of these parts are depicted in Fig.~\ref{fig:library-alloy-add-member-predicate}, which shows the predicate for ``Add Member". Any time a new objects is added to the domain model, the framework reduces the solution space by forcing a total order on the signatures for all classes and using the smallest one (line 2).

\begin{figure}[h]
\centering
\includegraphics[width=0.5\textwidth]{../Figures/library-alloy-add-member-predicate.png}
\caption{The predicate for the ``Add Member" use case.}
\label{fig:library-alloy-add-member-predicate}
\end{figure}

In addition to the corresponding predicate, the transformation produces a predicate with the prefix \textbf{fail\_}. This predicate is similar to the main predicate except that it does not define the post-conditions and the new objects, as it is only used to generate a state where the pre-conditions are falsified. This predicate can be used to test the system is capable of detecting when use cases are not allowed to run.

Additionally, it is possible to define different execution paths for a use case. In this case, an abstract signature is defined for the use case and for each execution path a concrete signature is defined. 

\subsection{Initial State}
\label{sec:alloy-initial-state}
For transforming the initial state model into an Alloy specification, the transformation specified the size of each relations. Also, for simplifying the solution space, the smallest possible object in the initial state is chosen using the total ordering. Fig.~\ref{fig:library-alloy-initial-state} shows the translation of the initial model in Fig.~\ref{fig:library-initial-state}. Lines 2-4 define a book that has been borrowed, lines 5-7 defines a member that can borrow books, and lines 8-10 define the association between these two objects.

\begin{figure}[h]
\centering
\includegraphics[width=0.5\textwidth]{../Figures/library-alloy-initial-state.png}
\caption{The initial state in Alloy.}
\label{fig:library-alloy-initial-state}
\end{figure}

\subsection{Labels Rules}
\label{sec:alloy-labels-rules}
The Alloy specification for the object labels rules are generated from our Java-based DSL introduced in Section~\ref{sec:subsidiary-models-object-labels}. Fig.~\ref{fig:library-alloy-labels-rules} shows the translation of the model in Fig.~\ref{fig:library-object-labels}.

\begin{figure}[h]
\centering
\includegraphics[width=0.5\textwidth]{../Figures/library-alloy-labels-rules.png}
\caption{The object labels rules in Alloy.}
\label{fig:library-alloy-labels-rules}
\end{figure}

\subsection{Input Partitioning}
\label{sec:alloy-input-partitioning}
The partitions of the input domain are also specified in Alloy. The specification for the input partitioning in the example library system, is depicted in Fig.~\ref{fig:library-alloy-input-partitions}. Each data type is modelled with an abstract signature (line 1) that is extended with a concrete signature for each partition (e.g.,\ line 2). If a partition has input parameters, they are also modelled (e.g.,\ line 5).

\begin{figure}[h]
\centering
\includegraphics[width=0.5\textwidth]{../Figures/library-alloy-input-partitions.png}
\caption{The input partitioning in Alloy.}
\label{fig:library-alloy-input-partitions}
\end{figure}


\subsection{Main Module and Test Goals}
\label{sec:alloy-main-module}
The main module is static and is independent of the SUT and its models. The module is depicted in Fig.~\ref{fig:alloy-main-module}. In the beginning, the module is defined and other modules are opened (lines 1-7). Then, the signatures for the system states and the abstract signature \texttt{Op}, used for tracing use cases, are defined (lines 9-12). Next, lines 15-17 require for all states, excluding the final state, a use case has to be executed. The predicate \texttt{apply} (given in Fig.~\ref{fig:alloy-main-apply}) decides which predicate (corresponding to the use cases), has to be applied to the model. Line 19 enforces the label rules and line 20 constricts the state space and forbids the creation of useless \texttt{Op} atoms. Lines 25-27 describe the initial state of the system. The search scope is defined in lines 29-30. Optional, user-defined constraints in the Alloy model can be added in the \texttt{testSystem} predicate. In this example, the search scope is 5 atoms for all signature, excluding atoms of the \texttt{State} signature (line 30). The scope for the \texttt{State} signature is 4, which results in execution traces of length three (one state is needed for the initial state).

\begin{figure}[h]
\centering
\includegraphics[width=0.5\textwidth]{../Figures/alloy-main-module.png}
\caption{The Alloy main module.}
\label{fig:alloy-main-module}
\end{figure}

\begin{figure}[h]
\centering
\includegraphics[width=0.5\textwidth]{../Figures/alloy-main-apply.png}
\caption{The \texttt{apply} predicate for the example library system.}
\label{fig:lalloy-main-apply}
\end{figure}

If the test goal is to generate exceptional states, line 21 is commented out and line 22 is uncommented. This applied the \texttt{fail\_apply} predicate to the final state. The predicate is similar to predicate \texttt{apply} except that it includes the negative predicates for the use cases. In this case, valid paths of length three are tested and then an exceptional operation is attempted at the final state. Exceptions cannot alter the domain model and therefore including them in the middle of a path is useless.

\subsection{Generated Test Cases}
\label{sec:alloy-test-cases}
For the example library system, Alloy generates 73 test cases with the search scope given in Fig.~\ref{fig:alloy-main-module}. One of the test cases is depicted in Appendix~\ref{app:library-testcases}, Fig.~\ref{fig:library-testcase}. The active use case in each state is shown by a diamond. The atom \texttt{System} is shown by gray rectangle and the atoms of the domain model are shown by bold rectangle. The labels and input data are respectively shown by eclipse and dashed lines.\todo{move to figure label or legend}

If the test goal is to generate exceptions (i.e.\ using the \texttt{fail\_apply} predicate), for the search scope of size one, five test cases are generated which are shown in Appendix~\ref{app:library-testcases}. Fig.~\ref{fig:library-exceptions-testcase-1} represents a test case that attempts to remove a member that has borowed a book. Fig.~\ref{fig:library-exceptions-testcase-2} shows a test case that attempts to deleted a book that has been borrowed. Fig.~\ref{fig:library-exceptions-testcase-3} illustrates an invalid borrow operation. Fig.~\ref{fig:library-exceptions-testcase-4} and Fig.~\ref{fig:library-exceptions-testcase-5} try to add a book with empty name and a book with a duplicate name (\texttt{Book0}), respectively.\todo{move to table}