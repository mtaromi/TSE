%
%		Library Example Test Case
%
\begin{figure*}[h]
\centering
{\includegraphics[width=0.9\textwidth]{../Figures/library-testcase-1}%
\label{fig:library-testcase-1}}
\hfil
{\includegraphics[width=0.9\textwidth]{../Figures/library-testcase-2}%
\label{fig:library-testcase-2}}
\caption{An execution path (test case) generated by Alloy.}
\label{fig:library-testcase}
\end{figure*} 

\begin{figure}[h]
\centering
\includegraphics[width=0.5\textwidth]{../Figures/library-exception-testcase-1}%
\caption{Exception state - remove member.}
\label{fig:library-exceptions-testcase-1}
\end{figure} 

\begin{figure}[h]
\centering
\includegraphics[width=0.5\textwidth]{../Figures/library-exception-testcase-2}%
\caption{Exception state - remove book.}
\label{fig:library-exceptions-testcase-2}
\end{figure} 

\begin{figure}[h]
\centering
\includegraphics[width=0.5\textwidth]{../Figures/library-exception-testcase-3}%
\caption{Exception state - borrow book.}
\label{fig:library-exceptions-testcase-3}
\end{figure} 

\begin{figure}[h]
\centering
\includegraphics[width=0.5\textwidth]{../Figures/library-exception-testcase-4}%
\caption{Exception state - add book: empty name.}
\label{fig:library-exceptions-testcase-4}
\end{figure} 

\begin{figure}[h]
\centering
\includegraphics[width=0.5\textwidth]{../Figures/library-exception-testcase-5}%
\caption{Exception state - add book: redundant name.}
\label{fig:library-exceptions-testcase-5}
\end{figure} 

