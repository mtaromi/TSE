
%% bare_jrnl_compsoc.tex
%% V1.4b
%% 2015/08/26
%% by Michael Shell
%% See:
%% http://www.michaelshell.org/
%% for current contact information.
%%
%% This is a skeleton file demonstrating the use of IEEEtran.cls
%% (requires IEEEtran.cls version 1.8b or later) with an IEEE
%% Computer Society journal paper.
%%
%% Support sites:
%% http://www.michaelshell.org/tex/ieeetran/
%% http://www.ctan.org/pkg/ieeetran
%% and
%% http://www.ieee.org/

%%*************************************************************************
%% Legal Notice:
%% This code is offered as-is without any warranty either expressed or
%% implied; without even the implied warranty of MERCHANTABILITY or
%% FITNESS FOR A PARTICULAR PURPOSE! 
%% User assumes all risk.
%% In no event shall the IEEE or any contributor to this code be liable for
%% any damages or losses, including, but not limited to, incidental,
%% consequential, or any other damages, resulting from the use or misuse
%% of any information contained here.
%%
%% All comments are the opinions of their respective authors and are not
%% necessarily endorsed by the IEEE.
%%
%% This work is distributed under the LaTeX Project Public License (LPPL)
%% ( http://www.latex-project.org/ ) version 1.3, and may be freely used,
%% distributed and modified. A copy of the LPPL, version 1.3, is included
%% in the base LaTeX documentation of all distributions of LaTeX released
%% 2003/12/01 or later.
%% Retain all contribution notices and credits.
%% ** Modified files should be clearly indicated as such, including  **
%% ** renaming them and changing author support contact information. **
%%*************************************************************************


% *** Authors should verify (and, if needed, correct) their LaTeX system  ***
% *** with the testflow diagnostic prior to trusting their LaTeX platform ***
% *** with production work. The IEEE's font choices and paper sizes can   ***
% *** trigger bugs that do not appear when using other class files.       ***                          ***
% The testflow support page is at:
% http://www.michaelshell.org/tex/testflow/


\documentclass[10pt,journal,compsoc]{IEEEtran}
%
% If IEEEtran.cls has not been installed into the LaTeX system files,
% manually specify the path to it like:
% \documentclass[10pt,journal,compsoc]{../sty/IEEEtran}





% Some very useful LaTeX packages include:
% (uncomment the ones you want to load)


% *** MISC UTILITY PACKAGES ***
%
%\usepackage{ifpdf}
% Heiko Oberdiek's ifpdf.sty is very useful if you need conditional
% compilation based on whether the output is pdf or dvi.
% usage:
% \ifpdf
%   % pdf code
% \else
%   % dvi code
% \fi
% The latest version of ifpdf.sty can be obtained from:
% http://www.ctan.org/pkg/ifpdf
% Also, note that IEEEtran.cls V1.7 and later provides a builtin
% \ifCLASSINFOpdf conditional that works the same way.
% When switching from latex to pdflatex and vice-versa, the compiler may
% have to be run twice to clear warning/error messages.






% *** CITATION PACKAGES ***
%
\ifCLASSOPTIONcompsoc
  % IEEE Computer Society needs nocompress option
  % requires cite.sty v4.0 or later (November 2003)
  \usepackage[nocompress]{cite}
\else
  % normal IEEE
  \usepackage{cite}
\fi
% cite.sty was written by Donald Arseneau
% V1.6 and later of IEEEtran pre-defines the format of the cite.sty package
% \cite{} output to follow that of the IEEE. Loading the cite package will
% result in citation numbers being automatically sorted and properly
% "compressed/ranged". e.g., [1], [9], [2], [7], [5], [6] without using
% cite.sty will become [1], [2], [5]--[7], [9] using cite.sty. cite.sty's
% \cite will automatically add leading space, if needed. Use cite.sty's
% noadjust option (cite.sty V3.8 and later) if you want to turn this off
% such as if a citation ever needs to be enclosed in parenthesis.
% cite.sty is already installed on most LaTeX systems. Be sure and use
% version 5.0 (2009-03-20) and later if using hyperref.sty.
% The latest version can be obtained at:
% http://www.ctan.org/pkg/cite
% The documentation is contained in the cite.sty file itself.
%
% Note that some packages require special options to format as the Computer
% Society requires. In particular, Computer Society  papers do not use
% compressed citation ranges as is done in typical IEEE papers
% (e.g., [1]-[4]). Instead, they list every citation separately in order
% (e.g., [1], [2], [3], [4]). To get the latter we need to load the cite
% package with the nocompress option which is supported by cite.sty v4.0
% and later. Note also the use of a CLASSOPTION conditional provided by
% IEEEtran.cls V1.7 and later.





% *** GRAPHICS RELATED PACKAGES ***
%
\ifCLASSINFOpdf
   \usepackage[pdftex]{graphicx}
  % declare the path(s) where your graphic files are
  % \graphicspath{{../pdf/}{../jpeg/}}
  % and their extensions so you won't have to specify these with
  % every instance of \includegraphics
  % \DeclareGraphicsExtensions{.pdf,.jpeg,.png}
\else
  % or other class option (dvipsone, dvipdf, if not using dvips). graphicx
  % will default to the driver specified in the system graphics.cfg if no
  % driver is specified.
  % \usepackage[dvips]{graphicx}
  % declare the path(s) where your graphic files are
  % \graphicspath{{../eps/}}
  % and their extensions so you won't have to specify these with
  % every instance of \includegraphics
  % \DeclareGraphicsExtensions{.eps}
\fi
% graphicx was written by David Carlisle and Sebastian Rahtz. It is
% required if you want graphics, photos, etc. graphicx.sty is already
% installed on most LaTeX systems. The latest version and documentation
% can be obtained at: 
% http://www.ctan.org/pkg/graphicx
% Another good source of documentation is "Using Imported Graphics in
% LaTeX2e" by Keith Reckdahl which can be found at:
% http://www.ctan.org/pkg/epslatex
%
% latex, and pdflatex in dvi mode, support graphics in encapsulated
% postscript (.eps) format. pdflatex in pdf mode supports graphics
% in .pdf, .jpeg, .png and .mps (metapost) formats. Users should ensure
% that all non-photo figures use a vector format (.eps, .pdf, .mps) and
% not a bitmapped formats (.jpeg, .png). The IEEE frowns on bitmapped formats
% which can result in "jaggedy"/blurry rendering of lines and letters as
% well as large increases in file sizes.
%
% You can find documentation about the pdfTeX application at:
% http://www.tug.org/applications/pdftex






% *** MATH PACKAGES ***
%
%\usepackage{amsmath}
% A popular package from the American Mathematical Society that provides
% many useful and powerful commands for dealing with mathematics.
%
% Note that the amsmath package sets \interdisplaylinepenalty to 10000
% thus preventing page breaks from occurring within multiline equations. Use:
%\interdisplaylinepenalty=2500
% after loading amsmath to restore such page breaks as IEEEtran.cls normally
% does. amsmath.sty is already installed on most LaTeX systems. The latest
% version and documentation can be obtained at:
% http://www.ctan.org/pkg/amsmath





% *** SPECIALIZED LIST PACKAGES ***
%
%\usepackage{algorithmic}
% algorithmic.sty was written by Peter Williams and Rogerio Brito.
% This package provides an algorithmic environment fo describing algorithms.
% You can use the algorithmic environment in-text or within a figure
% environment to provide for a floating algorithm. Do NOT use the algorithm
% floating environment provided by algorithm.sty (by the same authors) or
% algorithm2e.sty (by Christophe Fiorio) as the IEEE does not use dedicated
% algorithm float types and packages that provide these will not provide
% correct IEEE style captions. The latest version and documentation of
% algorithmic.sty can be obtained at:
% http://www.ctan.org/pkg/algorithms
% Also of interest may be the (relatively newer and more customizable)
% algorithmicx.sty package by Szasz Janos:
% http://www.ctan.org/pkg/algorithmicx




% *** ALIGNMENT PACKAGES ***
%
%\usepackage{array}
% Frank Mittelbach's and David Carlisle's array.sty patches and improves
% the standard LaTeX2e array and tabular environments to provide better
% appearance and additional user controls. As the default LaTeX2e table
% generation code is lacking to the point of almost being broken with
% respect to the quality of the end results, all users are strongly
% advised to use an enhanced (at the very least that provided by array.sty)
% set of table tools. array.sty is already installed on most systems. The
% latest version and documentation can be obtained at:
% http://www.ctan.org/pkg/array


% IEEEtran contains the IEEEeqnarray family of commands that can be used to
% generate multiline equations as well as matrices, tables, etc., of high
% quality.




% *** SUBFIGURE PACKAGES ***
%\ifCLASSOPTIONcompsoc
 \usepackage[caption=false,font=footnotesize,labelfont=sf,textfont=sf]{subfig}
%\else
%  \usepackage[caption=false,font=footnotesize]{subfig}
%\fi
% subfig.sty, written by Steven Douglas Cochran, is the modern replacement
% for subfigure.sty, the latter of which is no longer maintained and is
% incompatible with some LaTeX packages including fixltx2e. However,
% subfig.sty requires and automatically loads Axel Sommerfeldt's caption.sty
% which will override IEEEtran.cls' handling of captions and this will result
% in non-IEEE style figure/table captions. To prevent this problem, be sure
% and invoke subfig.sty's "caption=false" package option (available since
% subfig.sty version 1.3, 2005/06/28) as this is will preserve IEEEtran.cls
% handling of captions.
% Note that the Computer Society format requires a sans serif font rather
% than the serif font used in traditional IEEE formatting and thus the need
% to invoke different subfig.sty package options depending on whether
% compsoc mode has been enabled.
%
% The latest version and documentation of subfig.sty can be obtained at:
% http://www.ctan.org/pkg/subfig




% *** FLOAT PACKAGES ***
%
%\usepackage{fixltx2e}
% fixltx2e, the successor to the earlier fix2col.sty, was written by
% Frank Mittelbach and David Carlisle. This package corrects a few problems
% in the LaTeX2e kernel, the most notable of which is that in current
% LaTeX2e releases, the ordering of single and double column floats is not
% guaranteed to be preserved. Thus, an unpatched LaTeX2e can allow a
% single column figure to be placed prior to an earlier double column
% figure.
% Be aware that LaTeX2e kernels dated 2015 and later have fixltx2e.sty's
% corrections already built into the system in which case a warning will
% be issued if an attempt is made to load fixltx2e.sty as it is no longer
% needed.
% The latest version and documentation can be found at:
% http://www.ctan.org/pkg/fixltx2e


%\usepackage{stfloats}
% stfloats.sty was written by Sigitas Tolusis. This package gives LaTeX2e
% the ability to do double column floats at the bottom of the page as well
% as the top. (e.g., "\begin{figure*}[!b]" is not normally possible in
% LaTeX2e). It also provides a command:
%\fnbelowfloat
% to enable the placement of footnotes below bottom floats (the standard
% LaTeX2e kernel puts them above bottom floats). This is an invasive package
% which rewrites many portions of the LaTeX2e float routines. It may not work
% with other packages that modify the LaTeX2e float routines. The latest
% version and documentation can be obtained at:
% http://www.ctan.org/pkg/stfloats
% Do not use the stfloats baselinefloat ability as the IEEE does not allow
% \baselineskip to stretch. Authors submitting work to the IEEE should note
% that the IEEE rarely uses double column equations and that authors should try
% to avoid such use. Do not be tempted to use the cuted.sty or midfloat.sty
% packages (also by Sigitas Tolusis) as the IEEE does not format its papers in
% such ways.
% Do not attempt to use stfloats with fixltx2e as they are incompatible.
% Instead, use Morten Hogholm'a dblfloatfix which combines the features
% of both fixltx2e and stfloats:
%
% \usepackage{dblfloatfix}
% The latest version can be found at:
% http://www.ctan.org/pkg/dblfloatfix




%\ifCLASSOPTIONcaptionsoff
%  \usepackage[nomarkers]{endfloat}
% \let\MYoriglatexcaption\caption
% \renewcommand{\caption}[2][\relax]{\MYoriglatexcaption[#2]{#2}}
%\fi
% endfloat.sty was written by James Darrell McCauley, Jeff Goldberg and 
% Axel Sommerfeldt. This package may be useful when used in conjunction with 
% IEEEtran.cls'  captionsoff option. Some IEEE journals/societies require that
% submissions have lists of figures/tables at the end of the paper and that
% figures/tables without any captions are placed on a page by themselves at
% the end of the document. If needed, the draftcls IEEEtran class option or
% \CLASSINPUTbaselinestretch interface can be used to increase the line
% spacing as well. Be sure and use the nomarkers option of endfloat to
% prevent endfloat from "marking" where the figures would have been placed
% in the text. The two hack lines of code above are a slight modification of
% that suggested by in the endfloat docs (section 8.4.1) to ensure that
% the full captions always appear in the list of figures/tables - even if
% the user used the short optional argument of \caption[]{}.
% IEEE papers do not typically make use of \caption[]'s optional argument,
% so this should not be an issue. A similar trick can be used to disable
% captions of packages such as subfig.sty that lack options to turn off
% the subcaptions:
% For subfig.sty:
% \let\MYorigsubfloat\subfloat
% \renewcommand{\subfloat}[2][\relax]{\MYorigsubfloat[]{#2}}
% However, the above trick will not work if both optional arguments of
% the \subfloat command are used. Furthermore, there needs to be a
% description of each subfigure *somewhere* and endfloat does not add
% subfigure captions to its list of figures. Thus, the best approach is to
% avoid the use of subfigure captions (many IEEE journals avoid them anyway)
% and instead reference/explain all the subfigures within the main caption.
% The latest version of endfloat.sty and its documentation can obtained at:
% http://www.ctan.org/pkg/endfloat
%
% The IEEEtran \ifCLASSOPTIONcaptionsoff conditional can also be used
% later in the document, say, to conditionally put the References on a 
% page by themselves.




% *** PDF, URL AND HYPERLINK PACKAGES ***
%
\usepackage{url}
% url.sty was written by Donald Arseneau. It provides better support for
% handling and breaking URLs. url.sty is already installed on most LaTeX
% systems. The latest version and documentation can be obtained at:
% http://www.ctan.org/pkg/url
% Basically, \url{my_url_here}.





% *** Do not adjust lengths that control margins, column widths, etc. ***
% *** Do not use packages that alter fonts (such as pslatex).         ***
% There should be no need to do such things with IEEEtran.cls V1.6 and later.
% (Unless specifically asked to do so by the journal or conference you plan
% to submit to, of course. )


% correct bad hyphenation here
\hyphenation{op-tical net-works semi-conduc-tor}


\begin{document}
%
% paper title
% Titles are generally capitalized except for words such as a, an, and, as,
% at, but, by, for, in, nor, of, on, or, the, to and up, which are usually
% not capitalized unless they are the first or last word of the title.
% Linebreaks \\ can be used within to get better formatting as desired.
% Do not put math or special symbols in the title.
\title{A Lightweight Approach to Model-Based Acceptance Testing Using Alloy Specifications}
%
%
% author names and IEEE memberships
% note positions of commas and nonbreaking spaces ( ~ ) LaTeX will not break
% a structure at a ~ so this keeps an author's name from being broken across
% two lines.
% use \thanks{} to gain access to the first footnote area
% a separate \thanks must be used for each paragraph as LaTeX2e's \thanks
% was not built to handle multiple paragraphs
%
%
%\IEEEcompsocitemizethanks is a special \thanks that produces the bulleted
% lists the Computer Society journals use for "first footnote" author
% affiliations. Use \IEEEcompsocthanksitem which works much like \item
% for each affiliation group. When not in compsoc mode,
% \IEEEcompsocitemizethanks becomes like \thanks and
% \IEEEcompsocthanksitem becomes a line break with idention. This
% facilitates dual compilation, although admittedly the differences in the
% desired content of \author between the different types of papers makes a
% one-size-fits-all approach a daunting prospect. For instance, compsoc 
% journal papers have the author affiliations above the "Manuscript
% received ..."  text while in non-compsoc journals this is reversed. Sigh.

\author{Darioush~Jalalinasab,~
        Masoumeh~Taromirad,~
        and~Raman~Ramsin,~\IEEEmembership{Member,~IEEE}% <-this % stops a space
\IEEEcompsocitemizethanks{\IEEEcompsocthanksitem D.~Jalalinasab was with the Department of Computer Engineering, Sharif University of Technology, Tehran, Iran. Email: jalalinasab@ce.sharif.edu. \protect\\
% note need leading \protect in front of \\ to get a newline within \thanks as
% \\ is fragile and will error, could use \hfil\break instead.
\IEEEcompsocthanksitem M. Taromirad and R. Ramsin are with the Department of Computer Engineering, Sharif University of Technology, Tehran, Iran. Email: \{m.taromirad,ramsin\}@sharif.edu.}% <-this % stops an unwanted space
\thanks{Manuscript received July 23, 2018}}

% note the % following the last \IEEEmembership and also \thanks - 
% these prevent an unwanted space from occurring between the last author name
% and the end of the author line. i.e., if you had this:
% 
% \author{....lastname \thanks{...} \thanks{...} }
%                     ^------------^------------^----Do not want these spaces!
%
% a space would be appended to the last name and could cause every name on that
% line to be shifted left slightly. This is one of those "LaTeX things". For
% instance, "\textbf{A} \textbf{B}" will typeset as "A B" not "AB". To get
% "AB" then you have to do: "\textbf{A}\textbf{B}"
% \thanks is no different in this regard, so shield the last } of each \thanks
% that ends a line with a % and do not let a space in before the next \thanks.
% Spaces after \IEEEmembership other than the last one are OK (and needed) as
% you are supposed to have spaces between the names. For what it is worth,
% this is a minor point as most people would not even notice if the said evil
% space somehow managed to creep in.



% The paper headers
\markboth{IEEE TRANSACTIONS ON SOFTWARE ENGINEERING}%
{Shell \MakeLowercase{\textit{Jalalinasab et al.}}: Lightweight MBT Using Alloy}
% The only time the second header will appear is for the odd numbered pages
% after the title page when using the twoside option.
% 
% *** Note that you probably will NOT want to include the author's ***
% *** name in the headers of peer review papers.                   ***
% You can use \ifCLASSOPTIONpeerreview for conditional compilation here if
% you desire.



% The publisher's ID mark at the bottom of the page is less important with
% Computer Society journal papers as those publications place the marks
% outside of the main text columns and, therefore, unlike regular IEEE
% journals, the available text space is not reduced by their presence.
% If you want to put a publisher's ID mark on the page you can do it like
% this:
%\IEEEpubid{0000--0000/00\$00.00~\copyright~2015 IEEE}
% or like this to get the Computer Society new two part style.
%\IEEEpubid{\makebox[\columnwidth]{\hfill 0000--0000/00/\$00.00~\copyright~2015 IEEE}%
%\hspace{\columnsep}\makebox[\columnwidth]{Published by the IEEE Computer Society\hfill}}
% Remember, if you use this you must call \IEEEpubidadjcol in the second
% column for its text to clear the IEEEpubid mark (Computer Society jorunal
% papers don't need this extra clearance.)



% use for special paper notices
%\IEEEspecialpapernotice{(Invited Paper)}



% for Computer Society papers, we must declare the abstract and index terms
% PRIOR to the title within the \IEEEtitleabstractindextext IEEEtran
% command as these need to go into the title area created by \maketitle.
% As a general rule, do not put math, special symbols or citations
% in the abstract or keywords.
\IEEEtitleabstractindextext{%
\begin{abstract}
In the modern software era, testing plays an invaluable role in software development as a quality assurance measure. Estimates have been made that up to 50\% of effort and resources in software projects are allocated to testing. MBT takes a systematic route to test generation; this allows for a more goal-oriented and direct approach in achieving desirable test coverage criteria, especially when compared to ad-hoc testing. Keeping models updated has always been a dire problem in development approaches that are based on models, and MBT is no exception. Also, models that MBT requires, usually involve details that are not considered in normal development processes, or are completely different models than those developed during software analysis and design, which only makes this problem worse. The goal of this paper is to introduce a minimal set of structural and behavioral models that can be used as a basis for MBT. The primary concern in choosing these models was being intuitive and tangible for programmers and modelers. The paper introduces a lightweight approach to automated acceptance testing, and the related lightweight models and model transformations are elaborated upon. Finally, this approach was evaluated by application to a case study and comparison to other testing methods. Linear coverage and mutation testing were used as a part of the case study evaluation.
\end{abstract}

% Note that keywords are not normally used for peerreview papers.
\begin{IEEEkeywords}
Model Based Testing (MBT), Model Driven Development (MDD), Software Testing, Minimal Modelling, Static Analysis
\end{IEEEkeywords}}


% make the title area
\maketitle


% To allow for easy dual compilation without having to reenter the
% abstract/keywords data, the \IEEEtitleabstractindextext text will
% not be used in maketitle, but will appear (i.e., to be "transported")
% here as \IEEEdisplaynontitleabstractindextext when the compsoc 
% or transmag modes are not selected <OR> if conference mode is selected 
% - because all conference papers position the abstract like regular
% papers do.
\IEEEdisplaynontitleabstractindextext
% \IEEEdisplaynontitleabstractindextext has no effect when using
% compsoc or transmag under a non-conference mode.



% For peer review papers, you can put extra information on the cover
% page as needed:
% \ifCLASSOPTIONpeerreview
% \begin{center} \bfseries EDICS Category: 3-BBND \end{center}
% \fi
%
% For peerreview papers, this IEEEtran command inserts a page break and
% creates the second title. It will be ignored for other modes.
\IEEEpeerreviewmaketitle



\IEEEraisesectionheading{\section{Introduction}}
%
%		Introduction
%
\label{sec:introduction}
%\todo{update with changes from abstract}
\IEEEPARstart{T}{esting} plays an invaluable role in software development as a quality assurance measure, estimates show up testing uses up to 50\% of resources on projects~\cite{Beizer1990}. However, tests are costly to develop, and they can become complex and difficult to maintain~\cite{Farago2010a}. Unfortunately, determining what tests to write is a difficult task and is often left to the discretion and experience of the individual developer.

Automated Model-Based Testing (MBT) offers a promising solution to this problem~\cite{Tretmans2008}: automatically generating tests from high-level behavioural models side-steps the need to write and maintain tests manually. 
This generative approach to testing reduces the cost of creating and maintaining tests. The models are at a higher level of abstraction (making changes easier), and automated generation process obviates the need of making a single change in many test scripts. 
As an added benefit, MBT takes a systematic approach to test generation, so it is easier to determine what is covered more confidently.

However, practices prescribed by MBT are not particularly synergetic with those prescribed by lightweight software development processes~\cite{Farago2010a}, as they generally discourage the proliferation of models and dealing with the burden of keeping them up-to-date with code changes. 
MBT requires models different than those developed during general software analysis and design, or may require additional details not typically included (e.g.,\ describing properties in temporal logic languages~\cite{Tan2004}), thus making things seem worse. %\todo{needs examples for both}

This research introduces a minimalistic set of structural and behavioural models that can be used as a basis for MBT in the context of a lightweight software development process. The models are intuitive and accessible to practitioners: no particular modelling or model-based testing knowledge is required. These models are either normally produced during the development process, or have been kept very simple, if they are needed only for test generation.

Lightweight methodologies typically focus on unit testing, in part due to associated technical difficulties with acceptance testing (i.e, end to end testing)~\cite{Ambler2008,Ambler}. For example, statistics show that requirements specifications are still commonly captured in the form of documents. 
%\todo{this paragraph does not belong, move to related work?}
In this context, existing studies on the opportunity of benefiting from MBT in lightweight approaches and vice versa (e.g.,~\cite{Katara2006,Farago2010b,Loffler2010,Ussami2016}), by large, have focused on unit testing. Our approach focuses on acceptance testing.
A comprehensive study on the application of MBT in lightweight processes has been provided in our earlier work~\cite{Jalalinasab2012}. 

%\todo{citation needed, what are these difficulties}. 
%In this context, our approach focuses on acceptance testing.
%, and secondly, there are several unit testing proposals that can be directly used in such methodologies 
% (e.g.,~\cite{Beck2002,Farago2010a}). 
 

Automated acceptance testing techniques are generally either (1) script-based, or (2) based on system behavioural models. 
In \textit{script-based techniques}, developers convert informal requirements, one by one, from natural language to executable tests by writing test cases in a scripting language or library specified by a framework. The FIT~\footnote{FitNesse – \url{http://fitnesse.org}} and Cucumber~\footnote{Cucumber – \url{http://cukes.info}} frameworks, used in Acceptance Test Driven Development (ATDD)~\cite{Pugh2011} are examples of such techniques. This approach suffers from a low abstraction level and results in brittle, high-maintenance tests which must be developed and maintained individually. 
Techniques \textit{based on system behavioural models} generate tests from formal models of system behaviour, such as use cases (e.g.,~\cite{Nebut2006,Sarma2007,Kaplan2008}). These models implicitly define many valid execution paths for the system under test (SUT). %Complex models are usually requisited for these approaches, making them inappropriate for lightweight processes.
Deriving tests based on behavioural models addresses the abstraction level problem of script-based testing. However, the require models can be complex and inappropriate for lightweight processes.%\todo{too much repetition of this} 

In an aim to combine the best of both approaches, this paper presents an acceptance testing technique based on system behavioural models, using simple models, thus remaining practical for application in lightweight processes.%\todo{we go back and forth between agile/lightweight for no reason}. 
The proposed method uses class diagrams and use cases, that are considered as the most useful models in software development~\cite{Erickson2007,Erickson2008}\todo{why is this relevant?}. 

%The proposed approach is in fact an implementation of  the ``Directing testing by modelling the permissible order of operations" and %``Using static analysis" patterns that were introduced for applying MBT in agile/lightweight processes in our earlier %work~\cite{Jalalinasab2012}. \todo{not relevant, move to related work}

Fig.~\ref{fig:framework-structure} shows the overall structure of the proposed testing framework that has been tailored to data-oriented systems. Tests are generated from (1) a structural model of the domain (class diagram), (2) use cases, as the behavioural model, and (3) a static analyser (i.e.,\ Alloy~\cite{Jackson2002}). We use Alloy as a specification language~\cite{Jackson2002}, and the Alloy Analyser as the analysis engine. Alloy is a formal specification language based on first order logic, optimized for automated analysis.
The models are automatically translated into Alloy specifications. The solutions to the Alloy models are translated into executable test cases which the test driver and test harness components use to drive the SUT.

\begin{figure}[!t]
\centering
\includegraphics[width=0.5\textwidth]{../Figures/framework-structure.png}
\caption{Overall structure of the proposed framework.}
\label{fig:framework-structure}
\end{figure}

\textit{Contributions.} This paper makes the following
contributions:
\begin{itemize}
	\item \textit{Lightweight model-based acceptance testing.} We develop a lightweight testing framework with detailed description on how the approach is used from testers' perspective.
	
	\item \textit{Behavioural modelling DSLs.} We develop a set of simple DSLs for behavioural modelling.\todo{is this really a research contribution?}
	
	\item \textit{Implementation.} We provide the DSL used to create test models, the code that
	performs the model transformation to Alloy's native language, and the test driver and harness code necessary to execute test cases on the SUT.
	\todo{are we going to share the code, we have to.}

	\item \textit{Case-study. } We evaluated the proposed framework's applicability by demonstrating the feasibility of applying it to 
a medium-sized business application with reasonable resources.
We evaluated the effectiveness of the approach by comparing conventional test quality metrics (line coverage and mutation score)
to manually developed tests and also discuss bugs uncovered during the case study.

\end{itemize}

\textit{Outline.} The remainder of this paper is organized as follows.  %Section~\ref{sec:background} provides the background knowledge required to understand the contributions of our work.\todo{remove?}
Section~\ref{sec:running-example} explains the running example used to illustrate the testing framework. Section~\ref{sec:framework-overview} provides an overview of the proposed testing framework. Sections~\ref{sec:create-test-model} and~\ref{sec:test-generation-execution} describe the details of creating test models and test generation and execution, respectively. Section~\ref{sec:evaluation} presents the evaluation of the approach. Finally, the paper concludes with a discussion of limitations, and an outline of the related research and future work.


%\section{Background}
%%
% 	Background
%


\section{Running Example}
%
% 	Example
%
\label{sec:running-example}
To illustrate the framework and the testing activities, e.g., creating and preparing test models, we use a simple library system throughout the paper. 

In this system, a book has a unique name and there is only one copy of each book. Each member can borrow at most two books at any time. We identify (and use) six use cases for the system:
\begin{enumerate}
	\item Add Book 
	\begin{itemize}
		\item  Input: book name
		\item Pre-condition(s): the name should not be empty and redundant.
	\end{itemize}

	\item Add Member
	
	\item Borrow Book
	\begin{itemize}
		\item Pre-condition(s): the book should be free to borrow and the member has not exceeded the limitation.
	\end{itemize}

	\item Return Book
	
	\item Delete Book
	\begin{itemize}
		\item Pre-condition(s): the book should be free (not borrowed).
	\end{itemize}

	\item Delete Member
	\begin{itemize}
		\item Pre-condition(s): the member should not have any borrowed book.
	\end{itemize}
\end{enumerate}


\section{Framework Overview}
%
% 	Framework Overview
%
\label{sec:framework-overview}
Fig.~\ref{fig:testing-framework} shows an overview of the proposed testing framework.

From testers' perspective, the following steps are involved in using our framework:
\begin{enumerate}
	\item Developing a domain model is specified using EMF~\cite{EMF}. 
	
	\item Describing use cases with the provided DSL.
	
	\item Describing the initial state of the system are described with the given DSL.
	
	\item (Optionally) describe invariants of domain objects.
	
	\item Identify and optionally partitoin input data types using a provided DSL. 
	
	\item Specifying the test scope.
	
	\item Developing the test harness.
	
	\item All of the above models are automatically translated into Alloy's native modelling language.
	
	\item Alloy solves the model, yielding a set of execution traces. Each trace is a test cases and includes the required input values.
	
	\item The test driver runs executes each test case on the SUT using the test harness. The SUT is reset to the initial state
	before running the next test case.
\end{enumerate}


The \textit{primary models} required include use cases (translated from natural language to the specified DSL) and
the an EMF class diagram.
Additional \textit{subsidiary models} describe the initial state of the SUT, and optionally describes
invariants between object states.

The tester develops an SUT specific \textit{test harness} to mediate interaction with the test generation framework. The test harness consists of a \textit{data generator}, a \textit{command executor}, and an \textit{inspector}. \todo{??? describe these?}

\begin{figure*}[ht]
\centering
\includegraphics[width=\textwidth]{../Figures/testing-framework.png}
\caption{The architecture of the proposed testing framework.}
\label{fig:testing-framework}
\end{figure*}

Next, we discuss the components of the framework and the required models. 

\subsection{Structural Model}
\label{sec:framework-overview-structure}
The framework requires the tester to provide an EMF class diagram, conforming to Ecore metamodel.
This structural model includies domain entities essential to analysis. These entities typically represent the SUT's persistent data. 
In case the internal state of a domain object is relevant to the use cases, objects can optionally have a label.
An \textit{object label} is determined based on its internal data or the relations it has with other objects.
Labels determine whether an object is allowed to participate as input to use cases (see Section ~\ref{sec:create-test model-behaviour}). For example, an book may have a ``Borrowed" or ``Returned" label, and only \textit{returned} books can participate in the borrow  use case. Labels are similar to the ``State" pattern described in~\cite{Gamma1995}. 

Defining ``object labels" avoids exposing internal representations to the model finder (making analysis of the model computationally infeasible) and yet retains a degree of flexibility in a modelling and verifying behaviours of the SUT.

\subsection{Behavioural Model}
\label{sec:framework-overview-behaviour}
The tester develops a behavioural model of the SUT based on system use cases. This model describs effects of the use case on domain objects. Generally, use cases do not have the required formalism which makes them insufficient for automatic test generation~\cite{}. Our approach utilizes an internal domain specific language (DSL) within Java for describing use cases. This DSL provides the required formalism, yet remains familiar for the developer;  the behavioural model is constructed via chaining methods in a language that the developer is already familiar with (i.e.,\ Java).

Benefits of using a DSL as a modelling language include: 
\begin{itemize}
	\item developers are already familiar with the programming language. This allows existing IDEs and tools to be used for creating and maintaining the model,
	
	\item the use cases are modelled using textual format which is similar to programming and hence, creating behavioural model by our DSL would be familiar to developers (e.g.,\ by using familiar text editors), and 
	
	\item other testing frameworks such as JUnit~\cite{Beck2000}, jMock\footnote{ jMock – \url{http://jmock.org}}, and Selenium~\footnote{Selenium - \url{https://www.seleniumhq.org/}}, have been successful in using DSLs for specifying tests.
\end{itemize}

The introduced DSL supports modelling of
\begin{enumerate}
	\item input parameters for each use case,
	
	\item pre-conditions for executing a use case, 
	
	\item different execution paths for each use case, and
	
	\item the effects of use case execution on domain objects (post-conditions).
\end{enumerate}

Use cases are inherently not object-oriented, however we assume that mapping use cases to the domain model (e.g., mapping the input parameters and describing the effects of executing use cases) is relatively straightforward for data-intensive systems. Object labels can be used in pre- and post-conditions to help the tester gain some flexibility in this mapping.

\subsection{Test Data}
\label{sec:framework-overview-test-data}
Generating test data is one the main issues in automatic test generation~\cite{}. Input domain partitioning is a well known approach for generating test data~\cite{Ammann2008}. This approach partitions asks the tester to partitoin the input domain such that it is enough to test the system with only a single representative of every and each partition. This approach is particularly suited to the systems not involving complicated data manipulation and processing. 

The framework allows the tester to specify input data types and their partitioning in a Java based DSL. The tester then provides a data generator (as executable code) for each partitioning according to the architecture prescribed by the framework. 

\subsection{Subsidiary Models}
\label{sec:framework-overview-subsidiary}
In addition to the aforementioned models, the testing approach relies on three additional subsidiary models.

\subsubsection{Initial State} 
\label{sec:framework-overview-initial}
The initial state of the system is needed for model solving (static analysis). The objects present in the initial state and (optionally) their labels and relations are specified in our DSL.

\subsubsection{Object Labels Rules}
\label{sec:framework-overview-rules}
Certain constraints on relations between objects (e.g., number of involved objects in a relation) cannot be specified in the behavioural model DSL, as they do not pertain to a single use case. The framework includes a DSL for defining such dependencies  separately.

\subsubsection{Test Goals}
The tester specifies the test goal and the search scope (used in static analysis). There are two built-in test goals. Solutions obtained using the first goal are traces representing successful execution of use cases.
Solutions obtained using the second goal attempts to run use cases when the pre-conditions are not met, in an attempt to discover bugs.
The search scope determines the maximum length of execution traces, and is an inherit limitation to our apptoach, as we utilize a finite model generator (i.e.,\ Alloy).  The tester has the flexibility to force the inclusion or exclusion of a specific use case from execution traces. This may be useful, eg, to force a login in data-driven systems where most functionality is only accessible after logging in. Section~\ref{sec:alloy-main-module} provides more detail on how test goals and search scope are defined.


\subsection{Model Transformer: to Alloy}
\label{sec:framework-overview-model-transformer}
This component translates the structural and behavioural models into Alloy's native modelling language.


\subsection{Static Analyser: Alloy}
\label{sec:framework-overview-static-analyser}
The static analyser solves for a set of execution traces based on the the generated Alloy models. Alloy has a high-level specification language compared to CTL\footnote{CTL – Computation Tree Logic}, LTL\footnote{LTL – Linear Temporal Logic}, and first-order logic model analysis tools. The lower level of abstraction makes such languages not suitable targets for translating from high-level descriptions of system behavior.

Alloy has its own specification language for software specification which is built on the set theory and relational calculus. Most of  other approaches based on set theory (e.g., Z~\cite{Spivey1992}) have weak execution support. One advantage of using Alloy is its ability to enumerate all possible solutions for a given set of constraints for which it is also called a Model Finder. Alloy converts the specification to a SAT problem\footnote{Boolean SATisfiability problem}, which is then solved using a SAT solver. The solution (if exists) is mapped back to a relational model. Using SAT solvers gives rise to the need for a finite search space. This also limits the solver to finding small instances of the solution space. The ``Small Scope Hypothesis" claims ``most flaws in models can be illustrated by small instances, since they arise from some shape being handled incorrectly, and whether the shape belongs to a large or small instance makes no difference. So if the analysis considers all small instances, most flaws will be revealed"~\cite{Jackson2012}. This claim has not been proved, however, previous work has demonstrated the usefulness of approaches subject to this constraint in finding errors in software and models~\cite{??{.

\todo{move this to related works}
%In~\cite{Khorshid2004}, a method for generating tests for Java code using Alloy is introduced. This method works at the method level (single method) and Alloy specifications are written manually for method's pre- and post-conditions. Method's inputs are generated by Alloy based on pre-conditions, and after executing a method, the objects in the memory (heap) is compared with post-conditions (instances that alloy has found as solutions). The authors show that their method is effective for generating test cases throughout few case studies. 

\subsection{Test Driver}
\label{sec:framework-overview-test-driver}
The test driver uses the test harness to executes solutions to the Alloy model on the SUT.
Each solution is a test case that includes: (1) a number of use cases with their input parameters,
and (2) a description of the final expected \textit{state} of the SUT if the test case is run from the initial state.
The state of the SUT consists of a snapshot of the objects of the domain model at a particular moment. 

The driver instructs the harness to reset the SUT its initial state before running each test case.
The driver the uses the harness to run the use cases one by one (after generating required input parameters).
After the execution of each use case, the current state of the SUT is compared to the expected state of the system. 
Inconsistencies result in a failed test, and test execution stops. Otherwise, the next use case is executed.
Fig.~\ref{fig:test-driver-workflow} shows this workflow.

\begin{figure*}[!t]
\centering
\includegraphics[width=0.8\textwidth]{../Figures/test-driver-workflow.png}
\caption{The workflow of the test driver.}
\label{fig:test-driver-workflow}
\end{figure*}

\subsection{Test Harness}
\label{sec:framework-overview-test-harness}
The test harness is an SUT-specific piece of code developed by testers to utilise the testing framework. This component mediates all communications between the test driver and the SUT, including executing a use case and observing SUT behaviour. The test harness includes: 
\begin{enumerate}
	\item the \textit{command executor}, which is responsible for executing use cases on the SUT.  This component also includes machinery to generate representative data points from specified input partitions as needed. 
	
	\item the \textit{inspector}, which provides a list of domain objects and their labels to the test driver.
\end{enumerate}

\section{Creating Test Models}
%
% 	Creating Test Models
%
\label{sec:create-test-model}
The framework relies on the tester to provide the primary and subsidiary models. This section describes how these required models are created or generated in the context of the example library system. %It also covers some implementation aspects.?

\subsection{Structural Model}
\label{sec:create-test model-structure}
The structural model consists of a simplified EMF class diagram. Only the structure and relations are necessary, the internal data fields of classes are not modelled. Fig.~\ref{fig:library-structure-model} shows the structural model of the example library system.

\begin{figure}[h]
\centering
\subfloat[EMF Model]{\includegraphics[width=0.4\textwidth]{../Figures/library-class-diagram}%
\label{fig:library-emf}}
\hfil
\subfloat[Class Diagram]{\includegraphics[width=0.3\textwidth]{../Figures/library-emf}%
\label{fig:library-class}}
\caption{Structural model of the example library system.}
\label{fig:library-structure-model}
\end{figure} 

The Eclipse IDE\footnote{Eclipse - \url{http://eclipse.org}}, is capable of tranlating the diagram to Java code. This generated code is referenced in the DSL defining the behavioural model. The EMF model is also transformed into a partial Alloy specification (See  Section~\ref{sec:alloy-structure} for details of this transformation).

\subsection{Behavioural Model}
\label{sec:create-test model-behaviour}

The behavoiral model allows the tester to describe the effect of each use case on domain objects in terms of their labels and their relations. Labels can be identified by considering the state-based behaviour of the system, as implied by use cases. For example, in the library system, the class \texttt{Book} has two labels: \textit{Borrowed} and \textit{NotBorrowed}, and the class \texttt{Member} has three labels: \textit{HasNotBorrowed}, \textit{CanBorrow}, and \textit{CanNotBorrow}. Fig.~\ref{fig:library-object-labels} shows the labels and illustrates how they relate to use cases. Fig.~\ref{fig:library-usecase-borrow-book} shows the description of ``Borrow Book" use case using the DSL. 

\begin{figure}[h]
\centering
\subfloat[Member]{\includegraphics[width=0.4\textwidth]{../Figures/member-labels}%
\label{fig:library-member-labels}}
\hfil
\subfloat[Book]{\includegraphics[width=0.3\textwidth]{../Figures/book-labels}%
\label{fig:library-book-labels}}
\caption{State diagram showing the object labels in the library system.}
\label{fig:library-object-labels}
\end{figure} 

\begin{figure*}[h]
\centering
\includegraphics[width=0.7\textwidth]{../Figures/borrow-book.png}
\caption{The behavioural modelling for ``Borrow Book" use case using our DSL.}
\label{fig:library-usecase-borrow-book}
\end{figure*}

Each use case is defined with a Java class with a \texttt{@SystemOperation} annotation. A use case description specifies one or more execution paths that are defined as a method using the signature shown in line 3 of Fig.~\ref{fig:library-usecase-borrow-book}. These methods are annotated as \texttt{@Description} and return an \texttt{ModelExpectations} object. The core of the behavioural modelling is performed by configuring this (line 5 to 12). 

First (lines 5-7), the parameters and the pre-conditions of the use case are defined. 
Line 5 indicates the use case operates on an object of type \texttt{BOOK} which must have the \textt{NotBorrowed} label, and names it \texttt{thisBook}). This is done by chaining the \texttt{parameter} and \texttt{inState} methods. Lines 6 and 7 describe a parameter of type \texttt{Member} and its permissible labels in a similar fashion. All-caps class names directly reference the Java code generated by EMF. 

The second part (line 9 to 12) describes the post-conditions of the execution path, and includes changes to object labels.
A specific state (line 9) or an arbitrary state (line 10) may be specified. Changes in relations are also described (line 11). 
%Table~\ref{tbl:behavioural-modelling-preconditions} and Table~\ref{tbl:behavioural-modelling-postconditions}, in Appendix~\ref{app:behavioural-modelling}, list all the modelling features provided in the framework.

\subsection{Test Data}
\label{sec:create-test model-data}
This sections describes how data is generated for use cases that need input data. As mentioned in Section~\ref{sec:framework-overview-test-data}, we use the input partitioning approach for data generation. The partitioning of input parameters are defined by the tester, based on the system requirements, and are introduced to the framework. For example, the ``Add Book" use case requires the input parameter \textit{bookName} to should be automatically generated. A book name is a string has the into ``empty", ``duplicate", and ``unique" partitions.

\todo{this is way more detail than anyone would ever care about, also the code isn't that useful, can we cut this?}
In this framework, an input data type is an implementation of the \texttt{DataFactory} interface in which each partition is represented by a method annotated as \texttt{@Partition} and returns an object of type \texttt{PartitionDescription}. Each \texttt{PartitionDescription} is responsible for data generation for its respective partition, in its \texttt{generate} method. Fig.~\ref{fig:library-bookname} shows the definition of the \texttt{BookName} data type and its partitions, namely ``empty", ``unique", and ``nonUnique"\todo{does not match}. In lines 31-41, the partitioning is defined. Each partition extends  \texttt{PartitionDescription} and implements the \texttt{generate} method that is the main method for generating data for that partition. For example, lines 2-6 introduce the ``empty" partition that returns an \textit{empty} string in its \texttt{generate} method.

\begin{figure}[h]
\centering
{\includegraphics[width=0.5\textwidth]{../Figures/bookname-1}%
\label{fig:library-bookname-1}}
\hfil
{\includegraphics[width=0.5\textwidth]{../Figures/bookname-2}%
\label{fig:library-bookname-2}}
\caption{Defining \texttt{BookName} data type and its partitioning.}
\label{fig:library-bookname}
\end{figure} 

If the data generation requires access to the domain objects, this dependency is specified in the \texttt{getDataParams} method that returns a list of required objects. This list is then passed to the \texttt{generate} method, in addition to the \texttt{SoftwareSystem} object. The data generation for the ``nonUnique" partition, in Fig.~\ref{fig:library-bookname}, requires a book name that already exists in the system. This is done in lines 18-21 and the returned object that is used in the \texttt{generate} method, lines 24-28, as the generated (non-unique) book name. Fig.~\ref{fig:library-usecase-add-book} shows how a data type and its partitioning are used in modelling a use case. In line 4, calling the method \texttt{input} and referring to the \texttt{BookName} data type define a parameter for the use case. The method \texttt{inPartition} restricts the permissible partitions for that parameter.

\begin{figure*}[h]
\centering
\includegraphics[width=0.7\textwidth]{../Figures/add-book.png}
\caption{Using the data types and their partitioning in use case modelling.}
\label{fig:library-usecase-add-book}
\end{figure*}

\subsection{Test Harness}
\label{sec:create-test model-test-harness}

\subsubsection{Introducing the SUT}
\label{sec:test-harness-intro-SUT}
Testers implement the \texttt{SoftwareSystem} interface to introduce an SUT to the framework.
This object provides methods for executing commands and inspection. The \texttt{SoftwareSystem} interface includes a \texttt{reset} method, which is responsible for bringing the system into its initial state. This interface and its implementation for the example library system are shown in Fig.~\ref{fig:library-SUT}.\todo{doesn't really add value}

\begin{figure}[h]
\centering
\includegraphics[width=0.5\textwidth]{../Figures/library-SUT.png}
\caption{The \texttt{SoftwareSystem} interface and its implementation for example library system.}
\label{fig:library-SUT}
\end{figure}


\subsubsection{Inspection}
\label{sec:test-harness-inspection}
The test harness must return a list of the existing objects and their associated labels for each class in the domain model. Each object has a unique identifier (ID) managed by the test harness and the SUT. Accordingly, an abstract super class (\texttt{AbstractDomainObjInspector})  is provided that is inherited by the test harness. Fig.~\ref{fig:library-inspector} shows an example implementation of this super class and the inspector for the \texttt{Member} class.

\begin{figure}[h]
\centering
\includegraphics[width=0.5\textwidth]{../Figures/library-inspector.png}
\caption{An example inspector, for the ``Member" class.}
\label{fig:library-inspector}
\end{figure}

\todo{way too much detail..}
The \texttt{@Inspector} annotation with the name of the target \texttt{type} (line 1) allows the framework to identify the inspector for the target domain class (e.g., ``Member" in Fig.~\ref{fig:library-inspector}) . The constructor, line 5-7, takes an object implementing \texttt{SoftwareSystem}, which is the object representing the SUT (mentioned in Section~\ref{sec:test-harness-intro-SUT}). The \texttt{getObjectState} returns the labels of the object with the input ID, and the \texttt{getObjectList} returns the IDs of all the objects of the target type. The body of these two methods are implemented for the SUT.

\subsubsection{Command Execution}
\label{sec:test-harness-command-execution}
The test harness implements the \texttt{Command} interface (Fig.~\ref{fig:test-harness-command-interface}) for each use case so that it is possible to run the test cases on the SUT. Each use case description class has an \texttt{execute} method that returns an object of type \texttt{Command}. Calling this method would not execute the use case, it just creates a \texttt{Command} object which, later on, is used for test case execution by the framework during the test execution process.

\begin{figure}[h]
\centering
\includegraphics[width=0.5\textwidth]{../Figures/command-interface.png}
\caption{\texttt{Command} interface.}
\label{fig:test-harness-command-interface}
\end{figure}

An implementation of the \texttt{command} method, for the ``Borrow Book" use case, is shown in Fig.~\ref{fig:library-borrow-book-execute}, line 7-20. Considering the use case description in Fig.~\ref{fig:library-usecase-borrow-book}, the use case has two parameters which are passed to the \texttt{execute} method as objects of type \texttt{DomainParam}. These objects have a method, called \texttt{lookup}, which returns the ID of the object. The IDs are required in use case execution.

\begin{figure*}[h]
\centering
\includegraphics[width=0.7\textwidth]{../Figures/borrow-book-execute.png}
\caption{Implementation of the \texttt{Command} interface for ``Borrow Book" use case.}
\label{fig:library-borrow-book-execute}
\end{figure*}

Fig.~\ref{fig:library-usecase-add-book} shows how a data type is used in defining the implementation of the \texttt{execute} method (or creating the \texttt{command} object) for the ``Add Book" use case which requires an input parameter.  In line 8, the input parameter is passed to the \texttt{execute} method, as an object of type \texttt{InputParam} which has a \texttt{generate} method (line 12). Calling this method will call the \texttt{generate} method of the related partition and hence, results in generating appropriate test data. 

\subsection{Subsidiary Models}
\label{sec:create-test model-subsidiary-models}
This section elaborates on the subsidiary models introduced in Section~\section{XXX}.
The (user defined) test goals are directly specified Alloy models and thus not discussed further here.

\subsubsection{Initial State}
\label{sec:subsidiary-models-initial-state}
This model specifies the objects in the initial state, with their labels and the relations. Fig.~\ref{fig:library-initial-state} specifies an initial state for the example library system, consisting of one book and one member that has borrowed the book.

\begin{figure}[h]
\centering
\includegraphics[width=0.5\textwidth]{../Figures/library-initial-state.png}
\caption{An initial state model for the library system.}
\label{fig:library-initial-state}
\end{figure}

\subsubsection{Object Labels Rules}
\label{sec:subsidiary-models-object-labels}
Some constraints on relations (e.g., number of involved objects in a relation) cannot be specified in the behavioural model. For example, in the modelling of the post-conditions of the ``Borrow Book" use case (Fig.~\ref{fig:library-usecase-borrow-book}), the label of the object \texttt{thisMember} can not be determined a-priori to use case execution. 
Such dependencies are defined a subsidiary model. Fig.~\ref{fig:library-rules-member} illustrates the rules applicable to the ``Member" class and its labels. In line 3, the \texttt{setFor} method specifies the class the rules apply to. Line 4 indicates that if the given relation (\texttt{MEMBER\_BORROW}) is empty then the label is \texttt{HasNotBorrowed}. Similarly, line 5 says if the cadinality of the given relation (\texttt{MEMBER\_BORROW}) is equal or greater than two then the label is \texttt{CanNotBorrowed}. Line 6 indicates that the label would be \texttt{CanBorrow} if none of the other cases apply.

\begin{figure}[h]
\centering
\includegraphics[width=0.5\textwidth]{../Figures/library-rules-member.png}
\caption{Modelling of rules applied in ``Member Class".}
\label{fig:library-rules-member}
\end{figure}

\subsection{Introducing the Models to the Framework}
\label{sec:create-test model-introducing-models}
Each category of models described above is introduced to the framework by placing them in a package together.
The testing framework determines the name of this package by calling methods of the \textt{SUT} interface (Fig.~\ref{fig:SUT-interface}), then finds the required classes (and methods) using Java reflection.

\begin{figure}[h]
\centering
\includegraphics[width=0.4\textwidth]{../Figures/SUT-interface.png}
\caption{\texttt{SUT} interface.}
\label{fig:SUT-interface}
\end{figure}


\section{Test Generation and Execution}
%
% 	Farmework in Action: Test Generation and Execution
%
\label{sec:test-generation-execution}
The previous section introduces the required models and explains how to produce them. This section explains the test generation and execution workflow. \this~has been built upon Alloy~\cite{Jackson2002} and the Alloy Analyser (AA)~\cite{Jackson2000} with the aim of testing a system at the use case level (i.e.,\ acceptance testing). AA provides automatic analysis of Alloy specifications by generating instances that satisfy the constraints expressed in the specification.

The key idea behind \this~is to use Alloy to express the structural and behavioural model of the system specifying the use cases, the invariants of inputs and outputs, and pre-/post-conditions of executing use cases on the SUT. In this context, all the test models, introduced in the previous section, are automatically translated into Alloy specifications which are then used by AA to automatically generate all non-isomorphic~\cite{Shlyakhter2007} instances for that specification, for a given scope. Next, the testing framework translates these instances to real execution paths and concrete inputs, which form the executable test cases for the SUT.

In the following, we first describe the basics of the Alloy specification language and the Alloy Analyser; details can be found in~\cite{Jackson2000,Jackson2002,Jackson2012}. Then, we explain the detailed structure of generated Alloy models.

\subsection{Alloy}
\label{sec:test-generation-execution-alloy}
Alloy is a strongly typed language that assumes a universe of atoms partitioned into subsets, each of which has a basic type. An Alloy specification is a sequence of paragraphs that can be of two kinds: signatures, used for construction of new \textit{types}, and a variety of formula paragraphs, used to record \textit{constraints}. 

A \textit{signature} paragraph represents a basic (or uninterpreted) type and introduces an independent top-level set. A signature declaration may include a collection of relations (that are called \textit{fields}) along with the types of the fields and constraints on their values; a \textit{field} represents a set relation between the signatures. \textit{Formula} paragraphs are formed from Alloy expressions, specifying constraints of the desired solutions, such as \textit{facts} and \textit{assertions}. A \texttt{fact} is a formula that takes no arguments and need not be invoked explicitly; it is always true. An assertion (\texttt{assert}) is a formula whose correctness needs to be checked, assuming the facts in the model. 

The Alloy Analyser~\cite{Jackson2000} is an automatic tool for analysing models created in Alloy. Given a formula and a \textit{scope}--a bound on the number of atoms in the universe--the analyser determines whether there exists a model of the formula (that is, an assignment of values to the sets and relations that makes the formula true) that uses no more atoms than the scope permits, and if so, returns it. Since first order logic is undecidable, the analyser limits its analysis to a finite scope. The analysis is based on a translation to a boolean satisfaction problem, and gains its power by exploiting state-of-the-art SAT solvers. The models of formulae are termed \textit{instances} or \textit{solutions}. Each atom is in a signature and the relations between atoms are instances of their fields.

\subsection{Generated Alloy Modules}
\label{sec:test-generation-execution-alloy-models}
The test models described in Section~\ref{sec:framework-overview} are automatically translated into Alloy specifications consisting of six \texttt{modules}, that are defined in separate Alloy files (.als). These modules and their dependencies are shown in Fig.~\ref{fig:alloy-specifications}, and include:

\begin{enumerate}
	\item \texttt{system\_def\_auto}: translation of the domain structural model (EMF diagram),
	\item \texttt{operations\_def\_auto}: translation of the system behavioural model (use case descriptions) and definition of the labels,
	\item \texttt{rules\_def\_auto}: translation of the object labels rules,
	\item \texttt{init\_def\_auto}: translation of the initial state model, 
	\item \texttt{input\_def\_auto}: translation of the input partitionings, and
	\item \texttt{main}: the main module specifying the test goals and the entry point for the static analysis (i.e.,\ the starting predicate).
\end{enumerate}

\begin{figure}[h]
\centering
\includegraphics[width=0.4\textwidth]{../Figures/alloy-specifications.png}
\caption{Generated Alloy specifications and their dependencies.}
\label{fig:alloy-specifications}
\end{figure}

%The rest of this section describes the details of the model transformation for each module.

\subsection{Structure Model}
\label{sec:alloy-structure}
The EMF model of the system is automatically translated into Alloy, using the Xpand\footnote{Xpand – \url{http://eclipse.org/modeling/m2t/?project=xpand}} model-to-text transformation language. Fig.~\ref{fig:library-alloy-structure} shows the automatically generated Alloy specification for the structure of the example library system. Each class is translated into a signature. Each association is represented by a field in the respective signature (lines 9-11 and lines 17-19). The multiplicity of a relation affects the definition of the fields. Abstract signatures are used for modelling abstract classes. Class inheritance is implemented using the inheritance concept in Alloy.

\begin{figure}[h]
\centering
\includegraphics[width=0.5\textwidth]{../Figures/library-alloy-structure.png}
\caption{Alloy specification of the structure of the example library system.}
\label{fig:library-alloy-structure}
\end{figure}

%\subsubsection{State}
%\label{sec:alloy-state}
Alloy does not have a built-in notion of mutable state; it does not allow changes to the relations, whereas, in the context of testing, the domain model changes with the execution of use cases. Consequently, \this~models the concept of mutable state explicitly. A typical way to deal with this is defining a \texttt{state} signature and all the relations that would change during test executions are placed in a relation with an atom of \texttt{state}. The transformation adds the extension \texttt{-> State} at the end of the changing relations.

The \texttt{System} signature represents the system under test and tracks the objects that in the domain model at any time. The transformation defines a field for each concrete class in the domain model (e.g.,\ Fig.~\ref{fig:library-alloy-structure}, lines 1-4).

Object labels are modelled by the \texttt{DomainObjState} abstract signature, such that for each domain class, the transformation adds an abstract signature that extends \texttt{DomainObjState} and shows the labels for that class (e.g.,\ lines 15-19). Additionally, all the signatures for domain classes inherit from the \texttt{DomainObj} abstract signature which contains the field defining the relation between atoms and their labels.

In order to constrain the state space objects not in the domain model cannot participate in relations and can not have label atoms. This is specified as a \texttt{fact} at the end of the structural Alloy specification. The fact for the example library system is depicted in Fig.~\ref{fig:library-alloy-structure-fact}.

\begin{figure}[h]
\centering
\includegraphics[width=0.4\textwidth]{../Figures/library-alloy-structure-fact.png}
\caption{The \texttt{fact} for confining the state space in the structure model of the example library system.}
\label{fig:library-alloy-structure-fact}
\end{figure}


\subsection{Operations: Use Cases}
\label{sec:alloy-usecases}
The next step is to translate the use case descriptions into Alloy. Each use case is translated into a corresponding signature and a predicate. First, the signature is used for tracing the execution flow: a solution does not show which predicate in the specification has resulted in the solution. The test driver needs to know which predicate has been selected at any state. Using \textit{conjunction}, an atom of the signature is attached to each predicate, so that it is possible to trace the execution flow. The use case signature contains fields corresponding to its parameters. Fig.~\ref{fig:library-alloy-return-book-signature} shows the signature for the ``Return Book" use case. Lines 5-8 restrict the solution space.

\begin{figure}[h]
\centering
\includegraphics[width=0.5\textwidth]{../Figures/library-alloy-return-book-signature.png}
\caption{The signature for the ``Return Book" use case.}
\label{fig:library-alloy-return-book-signature}
\end{figure}

Each use case predicate takes atoms representing the system state before and after the use case is executed. An atom of the signature of the use case provides access to the inputs. This predicate essentially defines the constraints for creating the subsequent state. Fig.~\ref{fig:library-alloy-return-book-predicate} shows the predicate for ``Return Book". This predicate consists of the following parts: defining the domain of the input parameters, defining the pre-conditions, specifying the frame conditions, and defining the post-conditions.

\begin{figure}[h]
\centering
\includegraphics[width=0.5\textwidth]{../Figures/library-alloy-return-book-predicate.png}
\caption{The predicate for the ``Return Book" use case, describing the input parameters (lines 2-5), the pre-conditions (lines 6-10), the frame conditions (lines 11-20), and the post-conditions (lines 21-24).}
\label{fig:library-alloy-return-book-predicate}
\end{figure}

The predicate of a use case may also contains 1) a part for defining variables (using the \textbf{let} structure) and 2) a part for creating new objects, when new objects are added to the domain model. An example of these parts are depicted in Fig.~\ref{fig:library-alloy-add-member-predicate}, which shows the predicate for ``Add Member". Any time a new objects is added to the domain model, the framework reduces the solution space by forcing a total order on the signatures for all classes and using the smallest one (line 2).

\begin{figure}[h]
\centering
\includegraphics[width=0.5\textwidth]{../Figures/library-alloy-add-member-predicate.png}
\caption{The predicate for the ``Add Member" use case.}
\label{fig:library-alloy-add-member-predicate}
\end{figure}

In addition to the corresponding predicate, the transformation produces a predicate with the prefix \textbf{fail\_}. This predicate is similar to the main predicate except that it does not define the post-conditions and the new objects, as it is only used to generate a state where the pre-conditions are falsified. This predicate can be used to test the system for  detecting when use cases are not allowed to run.

Additionally, it is possible to define different execution paths for a use case. In this case, an abstract signature is defined for the use case and for each execution path a concrete signature is defined. 

\subsection{Initial State}
\label{sec:alloy-initial-state}
For transforming the initial state model into an Alloy specification, the transformation specified the size of each relations. Also, for simplifying the solution space, the smallest possible object in the initial state is chosen using the total ordering. Fig.~\ref{fig:library-alloy-initial-state} shows the translation of the initial model in Fig.~\ref{fig:library-initial-state}. Lines 2-4 define a book that has been borrowed, lines 5-7 defines a member that can borrow books, and lines 8-10 define the association between these two objects.

\begin{figure}[h]
\centering
\includegraphics[width=0.5\textwidth]{../Figures/library-alloy-initial-state.png}
\caption{The initial state in Alloy.}
\label{fig:library-alloy-initial-state}
\end{figure}

\subsection{Labels Rules}
\label{sec:alloy-labels-rules}
The Alloy specification for the object labels rules are generated from \this~Java-based DSL, introduced in Section~\ref{sec:subsidiary-models-object-labels}. Fig.~\ref{fig:library-alloy-labels-rules} shows the translation of the model in Fig.~\ref{fig:library-object-labels}.

\begin{figure}[h]
\centering
\includegraphics[width=0.5\textwidth]{../Figures/library-alloy-labels-rules.png}
\caption{The object labels rules in Alloy.}
\label{fig:library-alloy-labels-rules}
\end{figure}

\subsection{Input Partitioning}
\label{sec:alloy-input-partitioning}
The partitions of the input domain are also specified in Alloy. The specification for the input partitioning in the example library system, is depicted in Fig.~\ref{fig:library-alloy-input-partitions}. Each data type is modelled with an abstract signature (line 1) that is extended with a concrete signature for each partition (e.g.,\ line 2). If a partition has input parameters, they are also modelled (e.g.,\ line 5).

\begin{figure}[h]
\centering
\includegraphics[width=0.5\textwidth]{../Figures/library-alloy-input-partitions.png}
\caption{The input partitioning in Alloy.}
\label{fig:library-alloy-input-partitions}
\end{figure}


\subsection{Main Module and Test Goals}
\label{sec:alloy-main-module}
The main module is static and is independent of the SUT and its models. This module is depicted in Fig.~\ref{fig:alloy-main-module}. In the beginning, the module is defined and other modules are opened (lines 1-7). Then, the signatures for the system states and the abstract signature \texttt{Op}, used for tracing use cases, are defined (lines 9-12). Next, lines 15-17 require for all states, excluding the final state, a use case has to be executed. The predicate \texttt{apply} (given in Fig.~\ref{fig:alloy-main-apply}) decides which predicate (corresponding to the use cases), has to be applied to the model. Line 19 enforces the label rules and line 20 constricts the state space and forbids the creation of useless \texttt{Op} atoms. Lines 25-27 describe the initial state of the system. The search scope is defined in lines 29-30. Optional, user-defined constraints in the Alloy model can be added in the \texttt{testSystem} predicate. In this example, the search scope is 5 atoms for all signature, excluding atoms of the \texttt{State} signature (line 30). The scope for the \texttt{State} signature is 4, which results in execution traces of length three (one state is needed for the initial state).

\begin{figure}[h]
\centering
\includegraphics[width=0.5\textwidth]{../Figures/alloy-main-module.png}
\caption{The Alloy main module.}
\label{fig:alloy-main-module}
\end{figure}

\begin{figure}[h]
\centering
\includegraphics[width=0.5\textwidth]{../Figures/alloy-main-apply.png}
\caption{The \texttt{apply} predicate for the example library system.}
\label{fig:alloy-main-apply}
\end{figure}

If the test goal is to generate exceptional states, line 21 is commented out and line 22 is uncommented. This applied the \texttt{fail\_apply} predicate to the final state. The predicate is similar to predicate \texttt{apply} except that it includes the negative predicates for the use cases. In this case, valid paths of length three are tested and then an exceptional operation is attempted at the final state. Exceptions cannot alter the domain model and therefore including them in the middle of a path is useless.

\subsection{Generated Test Cases}
\label{sec:alloy-test-cases}
For the example library system, Alloy generates 73 test cases with the search scope given in Fig.~\ref{fig:alloy-main-module}, line 30 (i.e.,~\texttt{run testSystem for 5 but 4 State}). One of the test cases is depicted in Appendix~\ref{app:library-testcases}, Fig.~\ref{fig:library-testcase}. 

If the test goal is to generate exceptions (i.e.\ using the \texttt{fail\_apply} predicate), for the search scope of size one, five test cases are generated which are shown in Appendix~\ref{app:library-testcases}. Fig.~\ref{fig:library-exceptions-testcase-1} represents a test case that attempts to remove a member that has borowed a book. Fig.~\ref{fig:library-exceptions-testcase-2} shows a test case that attempts to deleted a book that has been borrowed. Fig.~\ref{fig:library-exceptions-testcase-3} illustrates an invalid borrow operation. Fig.~\ref{fig:library-exceptions-testcase-4} and Fig.~\ref{fig:library-exceptions-testcase-5} try to add a book with empty name and a book with a duplicate name (\texttt{Book0}), respectively.\todo{move to table}\todo{@Masoumeh: which table??}

\section{Evaluation}
%
% 	Case Study
%
\label{sec:evaluation}
To assess the effectiveness and applicability of our approach, we have first conducted a case study on a non-trivial medium-sized business application. Then, we evaluate the outcome based on different coverage criteria and mutation testing technique. 


\subsection{Case Study}
\label{sec:evaluation-case-study}
The subject is an organisation internal mailing system which is a windows-based desktop application. The system was developed throughout an iterative-incremental process using Java language. The code for the business logic consists of about 1000 LOC and the UI code has about 6000 LOC.
The requirements and use cases were documented in the analysis phase, and most of the use cases are about recording and manipulating the internal mails. Consequently, the system was appropriate for using our testing framework. Moreover, as the system was developed incrementally, the system was approved (and verified) by the customer at the end of each iteration, and  hence, the system has acceptable quality (i.e.,\ accepted by the customer).
The detail description of the case study is available in the master thesis of the first author~\cite{Jalalinasab2012b}. 

In the case study, we have focused on 15 use cases, depicted in Fig.~\ref{fig:case-use-cases}. The system has other use cases such as ``View Roles" and ``Aggregated Report", which do not make any change on the domain model and thus they were excluded in our study. 

\begin{figure*}[h]
\centering
\includegraphics[width=0.8\textwidth]{../Figures/case-use-cases.png}
\caption{Use case diagram for the case study.}
\label{fig:case-use-cases}
\end{figure*}

\textbf{Structural Model.} The structural model is shown in Fig.~\ref{fig:case-structure} which was created based on the analysis structural model created for the system in the analysis phase.

\begin{figure*}[!t]
\centering
%\subfloat[Class Diagram]{}
\includegraphics[width=0.9\textwidth]{../Figures/case-class-diagram.png}%
%\label{fig:case-class-diagram}}
%\hfil
%\subfloat[EMF Model]{\includegraphics[width=0.5\textwidth]{../Figures/case-emf-model.png}%
%\label{fig:case-emf-model}}
\caption{Structural model of the case study.}
\label{fig:case-structure}
\end{figure*} 

\textbf{Data Types.}  We have identified the following data types and partitioning for them:
\begin{itemize}
	\item Logical two-values; it is used for accepting or rejecting a process, and also defining the state of a mail, namely ``waiting" and ``for information". It has two partitions of ``true" and ``false".
	
	\item String; it is used for string inputs such as mail's content and title. It has two partitions of ``empty" and ``non-empty". 
	
	\item Role Name; for this data type three partitions, including ``empty", ``non-empty", and ``duplicate",  were defined.
	
	\item User Name: it has three partitions: ``empty", ``non-empty", and ``duplicate".
	
	\item Password Pairs: it is used for creating password in signing up. It has three partitions: ``empty", ``identical pairs", and ``non-identical pairs".
\end{itemize}

 \textbf{Object Labels.} Seven labels were identified and defined for the ``Mail" object, namely ``ForApproval", ``ApprovedBefore", ``Received", ``ReceivedNeedsReply", ``ReceivedBefore", ``Forwarded", and ``Sent". The ``Structure" domain object has also two labels including ``NotRoot" and ``HasRoot". Other domain objects do not have state-based behaviour and thus no labels were defined for them.

\textbf{Subsidiary Models.} Due to limited data dependencies, there is no rules for determining the labels, and thus there is no model of the rules. The only subsidiary model is the model describing the initial state. We used the initial state depicted in Fig.~\ref{fig:case-initial-state} in the testing experiment.

\begin{figure}[h]
\centering
\includegraphics[width=0.4\textwidth]{../Figures/case-initial-state.png}
\caption{Initial State for the case study.}
\label{fig:case-initial-state}
\end{figure}

\textbf{Test Harness.} The test harness was developed for our case study and it consists of about 300 LOC. For this case, we used the uispec4j\footnote{ uispec4 – http://uispec4j.org} in developing the components responsible for automatic communication with the UI. Additionally, most of the existing code were reused for developing the ``Inspector" component.
 
\textbf{Behavioural Model.} \label{sec:case-study-behaviour} Throughout modelling the use cases, we found out that there were sever analysis problems in three use cases, namely ``Remove User", ``Delete Role", and ``Remove Role", which stop the execution of the SUT and subsequently, the testing process. These use cases were not dealt with correctly; the related letters to a user or role, were not kept in the system in case of removing the user, deleting the role, or removing the role from the user, while it was explicitly specified in the requirements to keep the letter for other users and roles.
Consequently, we excluded these use cases from our study. The behavioural model included about 170 LOC in the internal DSL.  

\subsubsection{Test Execution and Result}
\label{sec:case-study-result}
After generating or defining the required models, the system was tested using the test goals and constrains, shown in Table~\ref{tbl:case-study-goals}. The test goals were defined incrementally starting with executing all short use cases and then longer paths covering complex use cases. We also tested the system in order to generate exceptions. All the computations were performed in one thread and each use case execution took about half a second, which could be improved by increasing the number of threads.	

\begin{table}[!t]
\caption{Test goals for the case study.}
\label{tbl:case-study-goals}
\centering
\begin{tabular}{|p{4cm}|c|c|}
\hline
Test Goal & \# Test Cases & Time (second)  \\ \hline
Execution paths of length 4 & 129 & 28 \\  \hline
Execution paths of length 6 including ``Send Internal Mail" & 72 & 26 \\ \hline
Execution paths of length 7 including ``Send Internal Mail" to non-lower staff & 180 & 79 \\ \hline
Execution paths of length 7 including ``Reply to Internal Mail" & 108 & 51 \\ \hline
Execution paths of length 8 including ``Confirm Internal Mail" & 360 & 214 \\ \hline
Execution paths of length 8 including ``Forward Internal Mail" & 288 & 174 \\ \hline\hline
Total & 1137 & 572 (10 mins) \\
\hline
\end{tabular}
\end{table}

Although the system was verified and approved by the customer throughout several iterations, we found out quite significant problems in different levels, including implementation, analysis, and requirement specification. For example, in the implementation level, we found out that the system generally accepts ``empty" strings  which, in some cases, is explicitly forbidden (e.g.,\ user name, password, and role name) and, in some cases, is unclear if it is permissible or not, due ambiguity in requirements (e.g.,\ content and title of a mail). We also found out that the system allows a role sends an internal mail to herself, while the system does not create any internal mail\footnote{The output of Alloy finding this error is available at \url{http://ce.sharif.edu/~jalalinasab/bug1.xml}} .

The problems in the analysis were explained in Sec~\ref{sec:case-study-behaviour}. In requirements specification, it was not clear if an internal mail to the direct upper manager needs to be approved or not\footnote{The output of Alloy finding this error is available at \url{http://ce.sharif.edu/~jalalinasab/bug4.xml}}. 

\subsection{Line Coverage}
\label{sec:case-study-line-coverage}
To evaluate the line coverage of our testing framework, we used EMMA\footnote{EMMA – http://emma.sourceforge.net/}, which manipulates the executable bytecode to find out which line is executed or not. The line coverage of the aforementioned test cases are shown in Table~\ref{tbl:case-study-coverage}. The ``Unrelated" refers to the classes for the use cases excluded in the study, which has not been specified for the business logic, as it was hard and impossible to clearly determine \textit{unrelated} code.

Excluding the ``Unrelated" code, the coverage for UI and business logic were \%90 and \%73 which results in total coverage of \%90, achieved without writing any test script, and is a high coverage for a non-trivial system.

\begin{table}[!t]
\caption{Line coverage for the case study.}
\label{tbl:case-study-coverage}
\centering
\begin{tabular}{|p{2cm}|c|c|c|c|}
\hline
Module & Covered & Not Covered & Unrelated & Total \\ \hline
UI & 3290 & 246 & 2310 & 5846 \\ \hline
Business Logic & 526	& 159 & - & 685 \\
\hline
\end{tabular}
\end{table}

\subsection{Mutation Testing}
\label{sec:case-study-mutation}
In order to assess the effectiveness of the proposed testing framework, mutation testing was used, in addition to line coverage. To do so, the PIT\footnote{PIT – \url{http://pitest.org/}} mutation generator tool was used. The main reason for choosing this tool was its applicability on large projects. The tool analyses the execution flow of use cases and accordingly, optimises the mutation generation processes, by determining proper mutation locations, prioritising the executions of use cases, and finding use cases for mutations. Another important feature of this tool is that the mutations are done on Java bytecode.

The default mutation operators supported in this tool were used in the case study, including 1) mutations in boundary values, 2) mutations by negation of conditions, 3) mutations in computational operators, 4) mutations in increment/decrement operators, 5) mutations in return values, and 6) mutations in methods with no return value.

The test cases used for line coverage (Table~\ref{tbl:case-study-goals}) were used for mutation testing, as well. The summary of the result is provided in Table~\ref{tbl:case-study-mutation-operators} and Table~\ref{tbl:case-study-mutation-modules}, showing the results respectively based on the mutation operators and the modules. Overally, \%56 of the mutants were killed while the mutations in the lines that were not covered by the test cases, have been considered. Accordingly, having these mutants excluded, the mutation score will be \%68 and \%79 respectively, for the UI and business logic modules. The low mutation score for the UI module is due to difficulties in identifying the method calls that are for updating the representation of the UI. 
However, the achieved mutation scores indicate that the generated test cases are able to distinguish between the main program and the acceptable number of mutants particularly where the code is covered by test cases.

\begin{table}[!t]
\caption{Mutation testing result with respect to the operators.}
\label{tbl:case-study-mutation-operators}
\centering
\begin{tabular}{|c|c|c|c|c|}
\hline
Operator & Survived & Killed & Not Covered & Total \\ \hline
1 & 2 & 21 & 6 & 29 \\ \hline
2 & 21 & 90 & 19 & 130  \\ \hline
3 & 3 & 4 & 0 & 7 \\ \hline
4 & 5 & 17 & 6 & 28 \\ \hline
5 & 14 & 75 & 28 & 117 \\ \hline
6 & 79 & 121 & 70 & 270 \\ \hline\hline
Total & 125 & 327 & 129 & 581 \\
\hline
\end{tabular}
\end{table}

\begin{table}[!t]
\caption{Mutation testing result with respect to the modules.}
\label{tbl:case-study-mutation-module}
\centering
\begin{tabular}{|c|c|c|c|c|}
\hline
Module & Survived & Killed & Not Covered & Total \\ \hline
UI & 87 & 186 & 57 & 330 \\ \hline
Business Logic & 38 & 141 & 72 & 251 \\
\hline
\end{tabular}
\end{table}

\subsection{Use Case Coverage}
\label{sec:case-study-usecase-coverage}
A very related coverage criteria to the testing framework is the \textit{use case coverage} which is classified into two types: 1) each use case has to be executed, at least, one time regardless of the execution paths inside the use case, and 2) all execution paths inside each use case have to be covered.

The first type can be easily achieved by the proposed framework via (i) putting \texttt{some [signature corresponding to the use case]}, for each use case, in the predicate defining the test goals, and (ii) increasing the search scope so that Alloy can find a solution. 

For the second type of use case coverage, the tester needs to define valid internal execution paths for each use case, in the behavioural model, and accordingly defines related signature for them. Then, Alloy Analyser is forced to cover specific paths by putting texttt{some [signature corresponding to the use case]\_[name of the execution path]} in the predicate defining the test goals, and incrementally increase the search scope.

Nevertheless, an execution path may imply different concrete (or actual) paths. For example, it may have an input with a partitioning which requires to choose an input for each of the partitions. Also, there could be various paths resulting in a failed use case. In all of these cases, the tester has to specifically, model the internal execution paths to achieve the coverage  based on the internal paths, as such the coverage is up to the paths specified by the tester which may however, not be included in the search space. In fact, an enough large search scope would result in complete coverage which is typically impossible in practice.

\subsection{Coverage based on Input Partitioning}
\label{sec:case-study-partitioning-coverage}
Another type of coverage criteria is based on the partitioning of the input domain. The testing framework creates all the permutations of the states and by choosing the two built-in test goals (complete execution of use cases and exceptions) the generated tests will cover all the state permutations. In fact, as the number of partitions are typically small, by choosing an enough large search scope all the permutations would be considered in the testing, i.e.,\ complete coverage.

Note that the labels of the domain objects also imply extra partitioning for the input domain and considering them as input partitioning would result in the challenging problem of determining the search scope.

\section{Discussion and Limitations}
%
% 	Discussion
%
\label{sec:discussion}
The presented testing framework has made few assumptions about the system under test and the methodology used for its development. Also, similar to any approach based on static analysis, our proposal is subject to constraints caused by using this technique. The assumptions and limitations are as follow.

\begin{itemize}
	\item The requirements are available and the use cases have detailed specification and contains the details required by the framework.
	
	\item The structural model of the primary and persistent classes of the domain is available.
	
	\item  The main target of the SUT is to manage the domain objects and hence, complex data processing systems are not suitable for using this framework.
	
	\item The state of the system is determined by the relationship between the domain objects and data dependencies are represented by object labels. In this context, the framework can not be used for the verification of protocols and reactive systems. 
	
	\item The system under test is deterministic.
	
	\item Execution of use cases do not have any side effect. Otherwise, the appropriate mechanisms have to be provided by the test harness.
	
	\item The mapping between the input parameters and the conditional statements, to the domain objects can be done easily.
\end{itemize}

Nevertheless, considering the target of this research that is testing data-oriented systems developed using lightweight development processes, these assumptions and limitations do not hamper (or limit) the application and benefits of the framework in practice. Our case study demonstrates the applicability and effectiveness of the proposed framework in testing a typical, non-trivial software application. It shows that, in addition to implementation errors, the framework would identify problems in analysis and ambiguities in requirements specifications, even throughout the development of a system that is continuously delivered and approved by the customer.

Additionally, considering the previous and similar studies in the literature, our framework advantages over them by 1) using lightweight and situational models, 2) automatically generating test execution paths, 3) providing flexible methods for generating test data and verification, and 4) providing traceability to requirements.

The work presented here is inspired by TestEra, presented by Khurshid et al.~\cite{Khurshid2004}, in which tests are generated using Alloy. Our work however, substantially differs from TestEra: 1) TestEra focuses on unit testing (tests are defined at method level) whereas in our work, use cases and their execution sequence are used for test generation (i.e.,\ acceptance testing), 2) in TestEra, the Alloy specifications are created manually and thus, the testers have to learn to model in Alloy, in contrast to the proposed framework, and 3) TestEra uses the heap memory and thus, is limited to executing time objects, while in our framework, the tester can define any approach for inspection.

Another similar work to ours is~\cite{Kaplan2008} in that tests are automatically generated based on the formal specification of requirements and the class diagram. The data types are limited to Integers and the verification is performed by a set of fault models to mutate an object diagram. Using a formal language, limiting to Integer data types, and verification based on fault models are the main differences between this work and our framework.

In~\cite{Scheetz1999}, a first-order logic language is used to define the system constraints, and the test goals are specified by object diagrams, both of which are not appropriate for using in lightweight processes--our main target. Their approach is mainly appropriate for systems that has complex state-based behaviour, in contrast to our framework.
Similarly,~\cite{Cavarra2002}, which mainly uses state diagrams for describing requirements, requires complex behavioural modelling which makes it unsuitable for lightweight development.

Testing based on ``virtual contracts"~\cite{Engels2006} is similar to our work as such they are analogous to pre-/post-conditions. But, they are defined for unit testing. Also, testers (developers) would be more familiar with our internal Java-based DSL in comparison to virtual contracts.

\section{Related Work}
%
% 	Related Work
%
\label{sec:related-work}
Here, we provide a discussion of the related efforts in the context of our research.

\subsection{Pre-/Post-conditions-based Testing}
Scheetz et al.~\cite{Scheetz1999} provides an approach for test generation based on class diagrams. The effects of each method on the state of the system (i.e.,\ post-conditions) are specified by a first order logic language. Test goals are the descriptions of the expected objects' states in the system that are defined manually. Test goals and the system initial state are represented by object diagrams. These models, in addition to the post-conditions of the methods are given to a \textit{planner} that generates a test suite that satisfies the test objectives. This approach is mainly appropriate for systems that has complex state-based behaviour, in contrast to our framework.

Cavarra et al.~\cite{Cavarra2002} present a test generation method based on class, object, and state diagrams. The class diagram identifies the entities in the system and the state diagrams--one for each class--explain how these entities may evolve. The object diagrams are used to describe the initial configuration of the system model, to specify a starting configuration in a test directive, and to flag configurations for inclusion or exclusion in a test model. A state diagram shows how an object will react to the arrival of an event. This method is similar to the method proposed in~\cite{Scheetz1999} except in that the behaviour is modelled by a first order logic language.

Bouquet et al.~\cite{Bouquet2007} defines a subset of UML 2.1~\cite{UML2} that allows formal behaviour models of the SUT. The subset uses class, instance and state diagrams, in addition to OCL~\cite{OCL} expressions. The models are used as input for a model-based test generator, called LEIRIOS Test Designer, that automates--using theorem prover--the generation of test sequences, covering each behaviour in the model. Object diagrams are used to specify the system initial state and test data. The state diagrams are an optional part and used to model the dynamic behaviour of the SUT as a finite state transition system. The OCL language has been extended to support execution semantics. For each test target, an automated theorem prover is used to search for a path from the initial state to that target, and to find data values that satisfy all the constraints along that path. 

\subsection{Static Analysis-based Testing}
The main idea in~\cite{Engels2006} to raise the abstraction level of the ``design by contract"~\cite{Meyer1992} concept to the level of design models. Each method's functionality is specified by a ``visual contract" --consists of two UML composite structure diagram. One diagram designates the pre-conditions, while the other specifies the post-conditions. Visual contracts are then automatically translated into JML\footnote{JML – Java Modeling Language – \url{http://jmlspecs.org}} assertions in the original code which are monitored, and an exception will be raised upon the violation of a contract. The static analysis techniques are used to automatically find methods' pre-conditions. The main shortcoming of this work is that it does not consider data dependencies. The work differs from the proposed framework in this paper as firstly,  it focus on unit testing and also it uses models for describing pre-/post-conditions. 

The technique presented in~\cite{Maoz2011} allows for QA on static aspects of class diagrams using static analysis. It proposes an extension to object diagrams, namely the ``modal object diagram" that allows defining positive/negative object configurations that the class diagram should allow/disallow. Doing so, each modal object diagram can be considered as a test case that the class diagram should satisfy. Ultimately, the verification is performed by a fully automated model checking-based technique.

The issue of testing UML models as independent from implementation is discussed in~\cite{Andrews2003}. It proposes several testing adequacy criteria for different UML models. The criteria for class diagrams involve creating instances with all possible multiplicities, creating all possible subclasses, and creating objects with different field combinations. A set of coverage criteria are introduced for UML collaboration diagrams including predicate and term coverage of UML guards assigned to message edges, and the execution all the messages and paths. These coverage can automatically be applied on the created models and subsequently result in set of test cases (the latter aspect, however, is not discussed in~\cite{Andrews2003}).


Di Nardo et al.~[Nardo,2017;Nardo, 2015a;Nardo, 2015b] focus on testing data processing software that requires generating complex data files or databases, while ensuring compliance with multiple constraints. The structure and the constraints of input data are modelled by UML class diagrams and OCL constraints. The approach is built upon UML2Alloy [Anastasakis , 2007] to generate an Alloy model that corresponds to the class diagram and the OCL constraints of the data model. The Alloy Analyser is used to generate valid instances of the data model to cover predefined fault types in a fault model.

%Alloy has been  successfully used to analyse Java programs [Jackson, 2000; Khurshid, 2000]. These techniques involve modelling  either the input data structures and computation in Alloy. %The manual translation of non-trivial imperative code into a declarative language is extremely subtle, and required careful thinking. In such approaches, modifications to an implementation require manual remodelling of computation.


\subsection{Specification-based Testing}
Specification-based testing has been intensively considered in the testing literature.  its importance was discussed in a very early paper by Goodenough and Gerhart~\cite{Goodenough1975}.

Different specifications have been considered and used for automatic test generation, such as  Z specifications~\cite{Spivey1992,Horcher1995,Stocks1996,Donat1997}, UML statecharts~\cite{Rumbaugh1999,Offutt1999}, ADL specifications~\cite{Sankar1994,Chang1999}, Alloy specifications~\cite{Jackson2002,Khurshid2004,Coppit2005}, JML specifications~\cite{Boyapati2002}.

Horcher~\cite{Horcher1995} presents a technique for software testing based on Z specifications~\cite{Spivey1992}. This technique provides automated test execution and result evaluation. However, concrete input test data need to be selected manually from an
automatically generated set of test classes.

The UMLTest tool [Offutt, 1999] automatically generates tests from UML statecharts and enabled transitions, but requires all variables to be boolean, among other limiting assumptions it makes about the UML input file. Applied to a C implementation of a cruise control, it detects several faults that were inserted by hand. 

Chang et al~\cite{Chang1999} present a technique for deriving test conditions--a set of boolean conditions on values of parameters--from ADL specifications~\cite{Sankar1994}. These test conditions are used to guide test selection and to measure comprehensiveness of existing test suites.

Boyapati et al.~\cite{Boyapati2002}, introduce Korat, a framework for automated testing of Java programs. Given a formal specification for a method in JML specifications, Korat uses the method pre-conditions and automatically generates all non-isomorphic test cases (within a given input size).The method post-conditions are used as a test oracle to check the correctness of each output.

Khurshid et al.~\cite{Khurshid2004} introduce a framework, called TestEra, for automated specification-based testing of Java programs. TestEra uses the method's pre-condition specification to generate test inputs and the post-condition to check correctness of outputs. TestEra supports specifications written in Alloy and uses the SAT-based back-end of the Alloy tool-set for systematic generation of test cases as JUnit test methods. Our work differs from TestEra for three main reasons: 1) TestEra focuses on unit testing (tests are defined at method level) whereas in our work, use cases and their execution sequence are used for test generation (i.e.,\ acceptance testing), 2) in TestEra, the Alloy specifications are created manually and thus, the testers have to learn to model in Alloy, in contrast to the proposed framework, and 3) TestEra uses the heap memory and thus, is limited to executing time objects, while in our framework, the tester can define any approach for inspection.

\cite{Kaplan2008} presents an approach for automatic test generation using the information provided in the domain and behavioural models of a system. The domain model consists of UML class diagram with invariants, while the behavioural model consists of UML use cases. Each use case flow has an associated guard condition and a set of updates (to the domain object diagram and the output parameters). The approach uses a formal language for modelling and is limited to integer data types. The verification is performed by a set of fault models to mutate an object diagram and a novel algorithm which distinguishes between the original and the mutated object diagrams (kind of conformance testing). Whereas, in our framework, the verification is done by the Inspector component and based on labels which is more flexible than the approach presented in~\cite{Kaplan2008}.

In a later work, Rosner et at.~\cite{Rosner2014} introduces \textsf{HyTeK}, a technique for bounded exhaustive test input generation that automatically generates test suites from input
specifications given in the form of hybrid invariants as such they  may be provided imperatively, declaratively, or as a combination of declarative and imperative predicates. \textsf{HyTeK} benefits  from optimization approaches of each side: (i) the information obtained while solving declarative portions of the invariant assists in pruning the search for partially valid structures from the imperative portion of the specification, and (ii) the \textit{tight bounds} are computed from the declarative invariant and used   during test generation both from the declarative and imperative parts of the specification, to reduce the search space. \textsf{HyTeK} combines a mechanism for processing imperative input specifications introduced in~\cite{Boyapati2002} through the Korat tool, with SAT solving for processing the declarative portions of the input specification, in the style put forward through the tool TestEra~\cite{Khurshid2004}.



\section{Conclusion}
%
% 	Conclusion
%
This paper presents a novel approach for model-based acceptance testing. It introduces a minimal set of structural and behavioural models that are normally produced during the development process, and also are intuitive and tangible for programmers and modellers. The domain model, represented as a class diagram or EMF model, is used as the structural model. Use cases, described by the provided DSLs, are used as the behavioural model. A use case description basically defines the pre-/post-conditions, and possibly the data for the use case.
Our approach employs static analysis for automatic test generation and execution. All the models are automatically translated into Alloy specifications which are solved by the Alloy Analyser resulted in valid correct execution traces for the SUT.

The proposal was evaluated throughout a case study on a medium-sized business application and analysing the result based on different coverage criteria, namely line coverage, use case coverage, and data coverage. We also applied mutation testing to assess the quality and effectiveness of the generated test cases. The study demonstrates that the framework provides acceptable line coverage (90\%) and mutation score of 72\%. In addition to implementation errors, the framework would help in identifying errors in analysis and requirements specification.

As for future work, we plan to improve the introduced DSLs to be more concise and provide more modelling features. Moreover, we plan to enhance model transformations to have optimised Alloy models in order to reduce test execution times. Finally, we would like to work on techniques to efficiently deal with evolution in test models.








% if have a single appendix:
%\appendix[Proof of the Zonklar Equations]
% or
%\appendix  % for no appendix heading
% do not use \section anymore after \appendix, only \section*
% is possibly needed

% use appendices with more than one appendix
% then use \section to start each appendix
% you must declare a \section before using any
% \subsection or using \label (\appendices by itself
% starts a section numbered zero.)
%


\appendices
%\section{Behavioural Modelling Features}
%\label{app:behavioural-modelling}
%%
%		Behavioural Modelling
%
\begin{table*}[!t]
% increase table row spacing, adjust to taste
%\renewcommand{\arraystretch}{1.3}
% if using array.sty, it might be a good idea to tweak the value of
% \extrarowheight as needed to properly center the text within the cells
\caption{Behavioural modelling: pre-conditions and parameters specification}
\label{tbl:behavioural-modelling-preconditions}
\centering
\begin{tabular}{|c||c|}
\hline
Signature & Description\\
\hline
\texttt{.parameter(``paramName", EMF\_TYPE)} & Define input parameter for a use case \\
\hline
\end{tabular}
\end{table*}

\begin{table*}[!t]
% increase table row spacing, adjust to taste
%\renewcommand{\arraystretch}{1.3}
% if using array.sty, it might be a good idea to tweak the value of
% \extrarowheight as needed to properly center the text within the cells
\caption{Behavioural modelling: post-conditions and expectations}
\label{tbl:behavioural-modelling-postconditions}
\centering
\begin{tabular}{|c||c|}
\hline
Signature & Description\\
\hline
\texttt{.expectNew(EMF\_TYPE, ``ValidState1", ``ValidState2", \ldots)} & Define input parameter for a use case \\
\hline
\end{tabular}
\end{table*}


% you can choose not to have a title for an appendix
% if you want by leaving the argument blank
\section{Case Study - Example Test Cases}
\label{app:library-testcases}
%
%		Library Example Test Case
%
\begin{figure*}[h]
\centering
{\includegraphics[width=0.9\textwidth]{../Figures/library-testcase-1}%
\label{fig:library-testcase-1}}
\hfil
{\includegraphics[width=0.9\textwidth]{../Figures/library-testcase-2}%
\label{fig:library-testcase-2}}
\caption{An execution path (test case) generated by Alloy.}
\label{fig:library-testcase}
\end{figure*} 

\begin{figure}[h]
\centering
\includegraphics[width=0.5\textwidth]{../Figures/library-exception-testcase-1}%
\caption{Exception state - remove member.}
\label{fig:library-exceptions-testcase-1}
\end{figure} 

\begin{figure}[h]
\centering
\includegraphics[width=0.5\textwidth]{../Figures/library-exception-testcase-2}%
\caption{Exception state - remove book.}
\label{fig:library-exceptions-testcase-2}
\end{figure} 

\begin{figure}[h]
\centering
\includegraphics[width=0.5\textwidth]{../Figures/library-exception-testcase-3}%
\caption{Exception state - borrow book.}
\label{fig:library-exceptions-testcase-3}
\end{figure} 

\begin{figure}[h]
\centering
\includegraphics[width=0.5\textwidth]{../Figures/library-exception-testcase-4}%
\caption{Exception state - add book: empty name.}
\label{fig:library-exceptions-testcase-4}
\end{figure} 

\begin{figure}[h]
\centering
\includegraphics[width=0.5\textwidth]{../Figures/library-exception-testcase-5}%
\caption{Exception state - add book: redundant name.}
\label{fig:library-exceptions-testcase-5}
\end{figure} 




% use section* for acknowledgment
\ifCLASSOPTIONcompsoc
  % The Computer Society usually uses the plural form
  \section*{Acknowledgments}
\else
  % regular IEEE prefers the singular form
  \section*{Acknowledgment}
\fi


This work was supported in part by Iran's National Elites Foundation. Raman Ramsin is the corresponding author.


% Can use something like this to put references on a page
% by themselves when using endfloat and the captionsoff option.
\ifCLASSOPTIONcaptionsoff
  \newpage
\fi



% trigger a \newpage just before the given reference
% number - used to balance the columns on the last page
% adjust value as needed - may need to be readjusted if
% the document is modified later
%\IEEEtriggeratref{8}
% The "triggered" command can be changed if desired:
%\IEEEtriggercmd{\enlargethispage{-5in}}

% references section

% can use a bibliography generated by BibTeX as a .bbl file
% BibTeX documentation can be easily obtained at:
% http://mirror.ctan.org/biblio/bibtex/contrib/doc/
% The IEEEtran BibTeX style support page is at:
% http://www.michaelshell.org/tex/ieeetran/bibtex/
\bibliographystyle{IEEEtran}
% argument is your BibTeX string definitions and bibliography database(s)
\bibliography{IEEEabrv,./ref}
%
% <OR> manually copy in the resultant .bbl file
% set second argument of \begin to the number of references
% (used to reserve space for the reference number labels box)
%\begin{thebibliography}{1}
%
%\bibitem{Jalalinasab2012}
%D.~Jalalinasab and R.~Ramsin, \emph{Towards Model-Based Testing Patterns for Enhancing Agile Methodologies}, in New Trends in Software Methodologies, Tools and Techniques – Proceedings of the Eleventh SoMeT'12, 2012, vol. 246, pp. 55–72.
%
%\end{thebibliography}

%\bibliographystyle{plain}
%\bibliography{../../../Resources/ref}

% biography section
% 

%\begin{IEEEbiography}[{\includegraphics[width=1in,height=1.25in,clip,keepaspectratio]{mshell}}]{Michael Shell}
% or if you just want to reserve a space for a photo:
%\newpage
\begin{IEEEbiography}{Darioush~Jalalinasab}
Biography text here.
\end{IEEEbiography}

% if you will not have a photo at all:
\begin{IEEEbiography}{Masoumeh~Taromirad}
Biography text here.
\end{IEEEbiography}

% insert where needed to balance the two columns on the last page with
% biographies
%\newpage

\begin{IEEEbiography}{Raman~Ramsin}
Biography text here.
\end{IEEEbiography}

% You can push biographies down or up by placing
% a \vfill before or after them. The appropriate
% use of \vfill depends on what kind of text is
% on the last page and whether or not the columns
% are being equalized.

%\vfill

% Can be used to pull up biographies so that the bottom of the last one
% is flush with the other column.
%\enlargethispage{-5in}



% that's all folks
\end{document}


% An example of a floating figure using the graphicx package.
% Note that \label must occur AFTER (or within) \caption.
% For figures, \caption should occur after the \includegraphics.
% Note that IEEEtran v1.7 and later has special internal code that
% is designed to preserve the operation of \label within \caption
% even when the captionsoff option is in effect. However, because
% of issues like this, it may be the safest practice to put all your
% \label just after \caption rather than within \caption{}.
%
% Reminder: the "draftcls" or "draftclsnofoot", not "draft", class
% option should be used if it is desired that the figures are to be
% displayed while in draft mode.
%
%\begin{figure}[!t]
%\centering
%\includegraphics[width=2.5in]{myfigure}
% where an .eps filename suffix will be assumed under latex, 
% and a .pdf suffix will be assumed for pdflatex; or what has been declared
% via \DeclareGraphicsExtensions.
%\caption{Simulation results for the network.}
%\label{fig_sim}
%\end{figure}

% Note that the IEEE typically puts floats only at the top, even when this
% results in a large percentage of a column being occupied by floats.
% However, the Computer Society has been known to put floats at the bottom.


% An example of a double column floating figure using two subfigures.
% (The subfig.sty package must be loaded for this to work.)
% The subfigure \label commands are set within each subfloat command,
% and the \label for the overall figure must come after \caption.
% \hfil is used as a separator to get equal spacing.
% Watch out that the combined width of all the subfigures on a 
% line do not exceed the text width or a line break will occur.
%
%\begin{figure*}[!t]
%\centering
%\subfloat[Case I]{\includegraphics[width=2.5in]{box}%
%\label{fig_first_case}}
%\hfil
%\subfloat[Case II]{\includegraphics[width=2.5in]{box}%
%\label{fig_second_case}}
%\caption{Simulation results for the network.}
%\label{fig_sim}
%\end{figure*}
%
% Note that often IEEE papers with subfigures do not employ subfigure
% captions (using the optional argument to \subfloat[]), but instead will
% reference/describe all of them (a), (b), etc., within the main caption.
% Be aware that for subfig.sty to generate the (a), (b), etc., subfigure
% labels, the optional argument to \subfloat must be present. If a
% subcaption is not desired, just leave its contents blank,
% e.g., \subfloat[].


% An example of a floating table. Note that, for IEEE style tables, the
% \caption command should come BEFORE the table and, given that table
% captions serve much like titles, are usually capitalized except for words
% such as a, an, and, as, at, but, by, for, in, nor, of, on, or, the, to
% and up, which are usually not capitalized unless they are the first or
% last word of the caption. Table text will default to \footnotesize as
% the IEEE normally uses this smaller font for tables.
% The \label must come after \caption as always.
%
%\begin{table}[!t]
%% increase table row spacing, adjust to taste
%\renewcommand{\arraystretch}{1.3}
% if using array.sty, it might be a good idea to tweak the value of
% \extrarowheight as needed to properly center the text within the cells
%\caption{An Example of a Table}
%\label{table_example}
%\centering
%% Some packages, such as MDW tools, offer better commands for making tables
%% than the plain LaTeX2e tabular which is used here.
%\begin{tabular}{|c||c|}
%\hline
%One & Two\\
%\hline
%Three & Four\\
%\hline
%\end{tabular}
%\end{table}


% Note that the IEEE does not put floats in the very first column
% - or typically anywhere on the first page for that matter. Also,
% in-text middle ("here") positioning is typically not used, but it
% is allowed and encouraged for Computer Society conferences (but
% not Computer Society journals). Most IEEE journals/conferences use
% top floats exclusively. 
% Note that, LaTeX2e, unlike IEEE journals/conferences, places
% footnotes above bottom floats. This can be corrected via the
% \fnbelowfloat command of the stfloats package.
