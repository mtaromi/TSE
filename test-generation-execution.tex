%
% 	Farmework in Action: Test Generation and Execution
%
\label{sec:test-generation-execution}
The previous section introduces the test models, required by the testing framework, and explaines how they are created or generated by a tester. This section explains the test generation and execution workflow. The testing framework has been built upon Alloy~\cite{Jackson2002} and the Alloy Analyser (AA)~\cite{Jackson2000} with the aim of testing a system at the use case level (i.e.,\ acceptance testing). AA provides automatic analysis of Alloy specifications by generating instances that satisfy the constraints expressed in the specification.

The key idea behind the framework is to use Alloy to express the structural and behavioural model of the system specifying the use cases, the invariants of inputs and outputs, and pre-/post-conditions of executing use cases on the system under test. In this context, all the test models, introduced in the previous section, are automatically translated into Alloy specifications which are then used by AA to automatically generate, for a given scope, all non-isomorphic~\cite{Shlyakhter2007} instances for that specification. Next, the testing framework uses reflection to translate these instances to real execution paths and concrete inputs, which form the executable test cases for the SUT.

In the following, we first describe the basics of the Alloy specification language and the Alloy Analyser; details can be found in~\cite{Jackson2000,Jackson2002,Jackson2012}. Then, we explain the model transformations into Alloy.

\subsection{Alloy}
\label{sec:test-generation-execution-alloy}
Alloy is a strongly typed language that assumes a universe of atoms partitioned into subsets, each of which is associated with a basic type. An Alloy specification is a sequence of paragraphs that can be of two kinds: signatures, used for construction of new \textit{types}, and a variety of formula paragraphs, used to record \textit{constraints}. Each specification starts with a \texttt{module} declaration that names the specification.%; existing specifications may be included in the current one using \texttt{open} declarations.

A \textit{signature} paragraph represents a basic (or uninterpreted) type and introduces an independent top-level sets. A signature declaration may include a collection of relations (that are called \textit{fields}) in it along with the types of the fields and constraints on their values; a \textit{field} represents a set relation between the signatures. \textit{Formula} paragraphs are formed from Alloy expressions, specifying the constraints on the desired solutions, such as \textit{facts} and \textit{assertions}. A \texttt{fact} is a formula that takes no arguments and need not be invoked explicitly; it is always true. An assertion (\texttt{assert}) is a formula whose correctness needs to be checked, assuming the facts in the model. 

Fig.~\ref{fig:alloy-example-list} shows a simple alloy model for Linked List. The model defines two signature: \texttt{List} and \texttt{Node}. The \texttt{List} signature has a field called \textit{header} showing the first element on the list. Line 9-12 define the facts on linked lists, indicating that all \textit{Nodes} are either the first element in a list or accessible by traversing the \texttt{next} field on a list. In this declaration, ``+" , ``.", and ``\^{}" are, respectively, operators for set union, relational composition, and transitive closure. Line 11 says that Linked Lists do not have a cycle. Lines 14-16 indicate that there is at least one Linked List in the solutions. Finally, in line 18, the main predicate used for finding the solutions, along with the search scope, is specified. 

\begin{figure}[h]
\centering
\includegraphics[width=0.4\textwidth]{../Figures/alloy-example-list.png}
\caption{Example Alloy model of Linked List.}
\label{fig:alloy-example-list}
\end{figure}

The Alloy Analyser~\cite{Jackson2000} is an automatic tool for analysing models created in Alloy. Given a formula and a \textit{scope}--a bound on the number of atoms in the universe--the analyser determines whether there exists a model of the formula (that is, an assignment of values to the sets and relations that makes the formula true) that uses no more atoms than the scope permits, and if so, returns it. Since first order logic is undecidable, the analyser limits its analysis to a finite scope. The analysis is based on a translation to a boolean satisfaction problem, and gains its power by exploiting state-of-the-art SAT solvers. The models of formulae are termed \textit{instances} or \textit{solutions}. For the example alloy model in Fig.~\ref{fig:alloy-example-list} with the given scope, there are seven solutions shown in Fig.~\ref{fig:alloy-example-list-solutions}. Each atom is in a signature and the relations between atoms are instances of their fields.

\begin{figure*}[h]
\centering
\includegraphics[width=\textwidth]{../Figures/alloy-example-list-solutions.png}
\caption{Solutions for the example Alloy model of Linked List.}
\label{fig:alloy-example-list-solutions}
\end{figure*}

\subsection{Generated Alloy Model}
\label{sec:test-generation-execution-alloy-models}
The test models, created/defined by testers, are automatically translated into Alloy specifications consisting of six \texttt{modules}, that are defined in separate Alloy files (.als). The modules and their dependencies are shown in Fig.~\ref{fig:alloy-specifications}, and include:

\begin{enumerate}
	\item \texttt{system\_def\_auto}: translation of the domain structural model (EMF diagram),
	\item \texttt{operations\_def\_auto}: translation of the system behavioural model (use case descriptions) and definition of the labels,
	\item \texttt{rules\_def\_auto}: translation of the object labels rules,
	\item \texttt{init\_def\_auto}: translation of the initial state model, 
	\item \texttt{input\_def\_auto}: translation of the input partitionings, and
	\item \texttt{main}: the main module specifying the test goals and the entry point for the static analysis (i.e.,\ the starting predicate).
\end{enumerate}

\begin{figure}[h]
\centering
\includegraphics[width=0.4\textwidth]{../Figures/alloy-specifications.png}
\caption{Generated Alloy specifications and their dependencies.}
\label{fig:alloy-specifications}
\end{figure}

In the rest of this section, the details of the model transformation for each module is described.

\subsection{Structure Model}
\label{sec:alloy-structure}
The EMF model of the system is automatically translated into an Alloy specification, using the Xpand\footnote{Xpand – \url{http://eclipse.org/modeling/m2t/?project=xpand}} model-to-text transformation language. Fig.~\ref{fig:library-alloy-structure} shows the automatically generated Alloy specification for the structure of the example library system. Each class is translated into a signature within which each association is represented by a field (lines 9-11 and lines 17-19). The multiplicity of a relation affects the definition of the fields. Abstract signatures are used for modelling abstract classes, and for modelling inheritance, the same concept in Alloy is used.

\begin{figure}[h]
\centering
\includegraphics[width=0.5\textwidth]{../Figures/library-alloy-structure.png}
\caption{Alloy specification of the structure of the example library system.}
\label{fig:library-alloy-structure}
\end{figure}

%\subsubsection{State}
%\label{sec:alloy-state}
Alloy does not have a built-in notion of state mutation; it does not allow to change the relationships, whereas, in the context of testing, the domain model would be changed because of executing the use cases. Consequently, the concept of changing state has to be explicitly defined in the Alloy specification. To this end, the \texttt{state} signature is defined and all the relations that would change during test executions, are in a relation with an atom of this signature. This is shown by putting the extension \texttt{-> State} at the end of the changing relations.

The \texttt{System} signature represents the system under test and keeps track of the objects that are in the domain model at any time. It defines a field for each concrete class in the domain model (e.g.,\ Fig.~\ref{fig:library-alloy-structure}, lines 1-4).

Object labels are modelled by the \texttt{DomainObjState} abstract signature, such that for each domain class, an abstract signature that extends \texttt{DomainObjState} and shows the labels for that class, is defined (e.g.,\ lines 15-19). Additionally, all the signature for the domain classes are inherited from the \texttt{DomainObj} abstract signature which contains the field defining the relation between atoms and their labels.

In order to confine the state space and prevent useless solutions, for testing, the objects that are not in the domain model can not be participated in relations and can not have label atoms. This is specified as a \texttt{fact} at the end of the structural Alloy specification. The fact for the example library system is depicted in Fig.~\ref{fig:library-alloy-structure-fact}.

\begin{figure}[h]
\centering
\includegraphics[width=0.4\textwidth]{../Figures/library-alloy-structure-fact.png}
\caption{The \texttt{fact} for confining the state space in the structure model of the example library system.}
\label{fig:library-alloy-structure-fact}
\end{figure}


\subsection{Operations: Use Cases}
\label{sec:alloy-usecases}
The next step is to translate the use case descriptions into Alloy. Basically, each use case is translated into a corresponding signature and a predicate. Firstly, the signature is used for tracing the execution flow: a solution does not show which predicate in the specification has resulted in the solution. Whereas, the test driver needs to know which predicate has been selected at any state. Thus, by using \textit{conjunction}, an atom of the signature is attached to each predicate, so that it is possible to trace the execution flow. Secondly, the use case signature consists of the fields corresponding to its parameters. The signature for the ``Return Book" use case is depicted in Fig.~\ref{fig:library-alloy-return-book-signature}. Lines 5-8 are for confining the state space.

\begin{figure}[h]
\centering
\includegraphics[width=0.5\textwidth]{../Figures/library-alloy-return-book-signature.png}
\caption{The signature for the ``Return Book" use case.}
\label{fig:library-alloy-return-book-signature}
\end{figure}

The corresponding predicate for each use case, takes the atoms of the previous and subsequent states, and an atom of the signature of the use case, and defines the constraints for creating the subsequent state. The predicate for the ``Return Book" use case is shown in Fig.~\ref{fig:library-alloy-return-book-predicate}. This predicate consists of the following parts: defining the domain of the input parameters (lines 2-5), defining pre-conditions (lines 6-10), specifying the frame conditions (lines 11-20), and defining the post-conditions (lines 21-24). 

\begin{figure}[h]
\centering
\includegraphics[width=0.5\textwidth]{../Figures/library-alloy-return-book-predicate.png}
\caption{The predicate for the ``Return Book" use case.}
\label{fig:library-alloy-return-book-predicate}
\end{figure}

The predicate of a use case may also contains 1) a part for defining variables (by using the \textbf{let} structure) and 2) a part for creating new objects, when new objects are added to the domain model. An example of these parts are depicted in Fig.~\ref{fig:library-alloy-add-member-predicate} showing the predicate for the ``Add Member" use case. New objects are chosen such that  choosing a new object does not result in different solutions by Alloy, which is done by defining a total order on the signatures for all classes. At any time, the smallest atom that does not exist in the domain is returned as the new object (line 2).

\begin{figure}[h]
\centering
\includegraphics[width=0.5\textwidth]{../Figures/library-alloy-add-member-predicate.png}
\caption{The predicate for the ``Add Member" use case.}
\label{fig:library-alloy-add-member-predicate}
\end{figure}

In addition to the corresponding predicate, a negative predicate with the prefix \textbf{fail\_} is generated in order to test the exceptions in the system. The predicate is similar to the main predicate except that it does not define the post-conditions and the new objects parts. It does not make any change in the domain model and only works on its pre-conditions, that are the negative of the original pre-conditions.

Additionally, it is possible to define different execution paths for a use case. In this case, an abstract signature is defined for the use case and for each execution path a concrete signature is defined. 

\subsection{Initial State}
\label{sec:alloy-initial-state}
For transforming the initial state model into an Alloy specification, the size of the fields are specified. Also, for simplifying the solution space, the smallest possible object in the initial state is defined. Fig.~\ref{fig:library-alloy-initial-state} shows the translation of the initial model in Fig.~\ref{fig:library-initial-state}. Lines 2-4 define a book that has been borrowed, lines 5-7 defines a member that can borrow books, and lines 8-10 define the association between these two objects.

\begin{figure}[h]
\centering
\includegraphics[width=0.5\textwidth]{../Figures/library-alloy-initial-state.png}
\caption{The initial state in Alloy.}
\label{fig:library-alloy-initial-state}
\end{figure}

\subsection{Labels Rules}
\label{sec:alloy-labels-rules}
The Alloy specification for the object labels rules are generated based on our internal Java-based DSL introduced in Section~\ref{sec:subsidiary-models-object-labels}. Fig.~\ref{fig:library-alloy-labels-rules} shows the translation of the model in Fig.~\ref{fig:library-object-labels}.

\begin{figure}[h]
\centering
\includegraphics[width=0.5\textwidth]{../Figures/library-alloy-labels-rules.png}
\caption{The object labels rules in Alloy.}
\label{fig:library-alloy-labels-rules}
\end{figure}

\subsection{Input Partitioning}
\label{sec:alloy-input-partitioning}
The partitionings of the input domain is also specified in Alloy. The specification for the input partitioning in the example library system, is depicted in Fig.~\ref{fig:library-alloy-input-partitions}. Each data type is modelled with an abstract signature (line 1) that is extended with a concrete signature for each partition (e.g.,\ line 2). If a partition has input parameters, they are also modelled (e.g.,\ line 5).

\begin{figure}[h]
\centering
\includegraphics[width=0.5\textwidth]{../Figures/library-alloy-input-partitions.png}
\caption{The input partitioning in Alloy.}
\label{fig:library-alloy-input-partitions}
\end{figure}


\subsection{Main Module and Test Goals}
\label{sec:alloy-main-module}
The main module is static and is independent of the SUT and its models. The module is depicted in Fig.~\ref{fig:alloy-main-module}. In the beginning, the module is defined and other modules are opened (lines 1-7). Then, the signatures for the system states and the abstract signature \texttt{Op}, used for tracing use cases, are defined (lines 9-12). Next, in line 15-17, it is indicated that in all the states, excluding the final state, a use case has to be executed. The predicate \texttt{apply} (given in Fig.~\ref{fig:alloy-main-apply}) decides which predicate, corresponding to the use cases, has to be applied on the model. Line 19 applies the rules on the labels and line 20 confines the state space and prevents from creating useless atoms \texttt{Op}. In lines 25-27, the initial state of the system is described, and the search scope is defined in lines 29-30. The additional, user-defined constraints in the Alloy model are defined in the predicate \texttt{testSystem}, if required. In this example, the search scope is 5 atoms for all signature, excluding the atoms for the \texttt{State} signature (line 30). The scope for the \texttt{State} signature is four that results in execution traces of length three (one state is four the initial state).

\begin{figure}[h]
\centering
\includegraphics[width=0.5\textwidth]{../Figures/alloy-main-module.png}
\caption{The Alloy main module.}
\label{fig:alloy-main-module}
\end{figure}

\begin{figure}[h]
\centering
\includegraphics[width=0.5\textwidth]{../Figures/alloy-main-apply.png}
\caption{The \texttt{apply} predicate for the example library system.}
\label{fig:lalloy-main-apply}
\end{figure}

If the test goal is to generate exceptional states, the line 21 is commented out and line 22 is uncommented, and thus, the predicate \texttt{fail\_apply} is applied on the final state. The predicate is similar to predicate \texttt{apply} except that it includes the negative predicates for the use cases. In this case, the true paths of length three are tested and then an exceptional state happens (final state). This is because that exceptions can not alter the domain model and hence, examining them in the middle of a path would be useless.

\subsection{Generated Test Cases}
\label{sec:alloy-test-cases}
For the example library system, the Alloy generates 73 test cases with the given search scope in Fig.~\ref{fig:alloy-main-module}. One of the test cases is depicted in Appendix~\ref{app:library-testcases}, Fig.~\ref{fig:library-testcase}. The active use case in each state is shown by diamond. The atom \texttt{System} is shown by gray rectangle and the atoms of the domain model are shown by bold rectangle. The labels and input data are respectively shown by eclipse and dashed lines.

If the test goal is to generate exceptions (i.e.\ using the \texttt{fail\_apply} predicate) for the search scope of size one, five test cases are generated which are shown in Appendix~\ref{app:library-testcases}. Fig.~\ref{fig:library-exceptions-testcase-1} shows the test case that removes a member that has a book in loan. Similarly, in Fig.~\ref{fig:library-exceptions-testcase-2}, a book that has been borrowed is deleted. Fig.~\ref{fig:library-exceptions-testcase-3} illustrates an invalid borrow operation. Fig.~\ref{fig:library-exceptions-testcase-4} and Fig.~\ref{fig:library-exceptions-testcase-5} tends to add a book with empty name and a book with a duplicate name (\texttt{Book0}), respectively.